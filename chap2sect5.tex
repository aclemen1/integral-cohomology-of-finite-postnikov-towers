\section{A counterexample towards transverse implications}\label{section:counterexample}

Let us consider the mod-$2$ cohomology Bockstein spectral sequence of $BSO$, the classifying space of the special orthogonal group. It is well known that 
$$
H^*(BSO;\F_2)\cong\F_2[w_j\ |\ j\geq2],
$$ where $\deg(w_j)=j$ for all $j\geq1$ (see \cite{Bo67} or \cite[p. 216]{Mc00}). By Wu's formulae (see for instance \cite[8, Part I, p. 138]{MT91}), it is also well known that 
$$
Sq^1w_j=
\begin{cases}
w_{j+1} &\text{if $j$ is even,}\\
0 &\text{otherwise.}
\end{cases}
$$ These informations are sufficient to deduce the entire Bockstein spectral sequence.\index{Bockstein spectral sequence of $BSO$} Actually, we have
\begin{align*}
B_1^*&\cong \F_2[w_j\ |\ j\geq2],\\
B_2^*&\cong \F_2[w_{2j}^2\ |\ j\geq1],\\
B_3^*&\cong B_2^*\text{, since $d_2$ is trivial for dimension reasons,}\\
&\vdots\\
B_\infty^*&\cong B_2^*\cong \F_2[w_{2j}^2\ |\ j\geq1].
\end{align*}
Let us consider the lowest degree element $w_2$ which is obviously primitive and $0$-transverse since $Sq^1w_2=w_3$ by Wu's formulae. We have $d_2w_2^2=0$ and therefore $w_2$ is not $1$-transverse. This shows that the hypothesis of Theorem \ref{t:transversity for Eilenberg-MacLane spaces} are not sufficient for a generalization to all simply connected H-spaces.


\endinput