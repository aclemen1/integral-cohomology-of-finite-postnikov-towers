\chapter{A universal table for admissible words}

The purpose of this Appendix is to establish a universal table in which we can read all the admissible words in given stable degree and height range. This is an extension of results obtained in Chapter \ref{chapter:2}. Results are stated without proof.

\section{Elementary theoretical results}\label{appendix:elementary results}

\begin{defn}\index{height}
The {\bf height} is a map $h:{\mathcal S}\to\N$ defined for all $I=(a_0,\dots,a_k)\in{\mathcal S}$ by
$$
h(I)=\begin{cases}
e(I)+1 &\text{if $a_0$ is even and}\\
e(I) &\text{if $a_0$ is odd.}
\end{cases}
$$
Let ${\mathcal S}_{q,n}$ be the graded subset given by all admissible sequences $I\in{\mathcal S}$ such that $\degst(I)=q$ and $h(I)\leq n$. Let ${\mathcal S}_{*,n}=\cup_{q\geq0}{\mathcal S}_{q,n}$.
\end{defn}

\begin{defn}\label{Ad:g}\index{0@$g_I(n)$}
Let $n\geq0$. For all admissible sequence $I=(a_0,\dots,a_k)\in{\mathcal S}$ with $e(I)\leq n$ we define an admissible word $g_I(n)\in{\mathcal W}$ inductively as follows:
\begin{align*}
g_{(0)}(n)&=\sigma^n,\\
g_{(1)}(n)&=\sigma^{n-1}\psi_2,\\
g_I(n)&=\begin{cases}
\beta_2\sigma^{n-e(I)-1}\varphi_2 g_{I^-}(e(I)) &\text{if $a_0\equiv0(2)$ and $e(I)<n$,}\\
\gamma_2g_{I^-}(n) &\text{if $a_0\equiv0(2)$ and $e(I)=n$,}\\
\sigma^{n-e(I)}\varphi_2 g_{I^-}(e(I)-1) &\text{if $a_0\equiv1(2)$,}\\
\end{cases}\\
&\qquad\text{when $I\not=(0)$ and $(1)$.}
\end{align*}
Moreover we set $g_I=g_I(h(I))$.
\end{defn}

\begin{prop}
The map $g_{(-)}:{\mathcal S}\to{\mathcal W}$ preserves heights and stable degrees.
\end{prop}

\begin{prop}\label{Ap:reduction}
We have $g_I(n)=\sigma^{n-h(I)}g_I$.
\end{prop}

\begin{prop}\label{Ap:completeness}
We have ${\mathcal W}-\{\sigma^{n}\psi_{2^s}\ |\ \text{$n\geq0$ and $s\geq2$}\}=\{g_I(n)\ |\ \text{$I\in{\mathcal S}$ and $n\geq0$}\}$.
\end{prop}
\newpage

\section{Universal table in range $1\leq h\leq5$, $0\leq \degst\leq10$}\index{universal table}

To obtain all admissible words $\mathcal W$, it suffices to apply Proposition \ref{Ap:completeness} and therefore to consider Proposition \ref{Ap:reduction} and the set $\{g_I\ |\ I\in{\mathcal S}\}$. Let us put this set in a table where we have substituted the only admissible word of height $1$ and stable degree $1$, $\psi_2$, by $\psi_{2^s}$.

\begin{align*}
\tiny
\xymatrix@R=0.4cm@C=0.25cm{
\ar@{-}[rd]^h_{\degst}
&*+[.]{}\ar@{-}[ddddddddddddd]&1 &2 &3 &4 &5\\
*+[.]{}\ar@{-}[rrrrrrr] &*+[.]{} &&&&&&\\
0 &&\sigma\ar@{-}[d]_<{(0)}^>{(1)}\\
1 &&\psi_{2^s}\\
2 &&&&\beta_2\varphi_2\sigma^2\ar@{-}[d]_<{(2)}^>{(3)}\\
3 &&&\beta_2\varphi_2\psi_2\ar@{-}[d]_<{(2,1)}^>{(3,1)} &\varphi_2\sigma^2\\
4 &&&\varphi_2\psi_2 &&&\beta_2\varphi_2\sigma^4\ar@{-}[d]_<{(4)}^>{(5)}\\
5 &&&&&\beta_2\varphi_2\sigma^2\psi_2\ar@{-}[d]_<{(4,1)}^>{(5,1)} &\varphi_2\sigma^4\\
6 &&&&\beta_2\varphi_2\gamma_2\sigma^2\ar@{-}[d]_<{(4,2)}^>{(5,2)} &\varphi_2\sigma^2\psi_2\\
7 &&&\beta_2\varphi_2\gamma_2\psi_2\ar@{-}[d]_<{(4,2,1)}^>{(5,2,1)} &\varphi_2\gamma_2\sigma^2\\
8 &&&\varphi_2\gamma_2\psi_2 &&&\beta_2\varphi_2\sigma^2\gamma_2\sigma^2\ar@{-}[d]_<{(6,2)}^>{(7,2)}\\
9 &&&&&\beta_2\varphi_2\sigma^2\gamma_2\psi_2\ar@<-7ex>@{-}[d]_<{(6,2,1)}^>{(7,2,1)}\qquad \beta_2\varphi_2^2\sigma^2\ar@<8ex>@{-}[d]_<{(6,3)}^>{(7,3)} &\varphi_2\sigma^2\gamma_2\sigma^2\\
10 &&&&&\varphi_2\sigma^2\gamma_2\psi_2\qquad \varphi_2^2\sigma^2\\
&
}
\end{align*}

Vertical lines show that admissible words are paired off by a Bockstein operation (eventually ``higher'' Bockstein operation in the case $(\sigma,\psi_{2^s})$ when $s\geq2$). Near all admissible words stand the correponding admissible sequences.

It may be interesting to remark that more than one admissible word can occur in given stable degree and height. For instance, both $\beta_2\varphi_2\sigma^2\gamma_2\psi_2$ and $\beta_2\varphi_2^2\sigma^2$ have stable degrees equal to $9$ and heights equal to $4$.
\newpage

\section{User's guide and examples}

Suppose that you want to determine all admissible words of height $h$ and degree $d$. Then consider in the line $\degst=d-h$ all the admissible words of height $\leq h$. For each of these words, add $\sigma^k$ on the left with $k\geq0$ such that the degree of the new admissible word is exactly $d$.

\begin{exmp*}[Admissible words of height $3$ and degree $6$]
In the line $\degst=6-3=3$, we must consider the words of height $\leq 3$, namely $\beta_2\varphi_2\psi_2$ and $\varphi_2\sigma^2$, which are of degree $5$ and $6$ respectively. Thus admissible words of height $3$ and degree $6$ are $\sigma\beta_2\varphi_2\psi_2\sim\beta_2\sigma\varphi_2\psi_2$ and $\varphi_2\sigma^2$.
\end{exmp*}

A glance at the table of admissible words involved in the calculus of $H_*(K(\Z/2,3);\Z)$ at p. \pageref{subsection:Admissible Words Involved in the Calculus of K(Z/2,3)} also shows that these words $\beta_2\sigma\varphi_2\psi_2$ and $\varphi_2\sigma^2$ are the only two words of height $3$ and degree $6$.\bigskip

Suppose now that you want to determine $H_*(K(\Z/2^s,n);\F_2)$ as a graded $\F_2$-vector space. Recall that following Theorem \ref{t:Cartan's description in homology}, we must determine all admissible words of height $\leq n$.

\begin{exmp*}[$H_{\leq10}(K(\Z/2,2);\F_2)$]
Admissible words of height $2$ and stable degree $\leq8$ (i.e. degree $\leq10$) are $\sigma^2$ ($d=2$), $\sigma\psi_2$ ($d=3$), $\beta_2\varphi_2\psi_2$ ($d=5$), $\varphi_2\psi_2$ ($d=6$), $\beta_2\varphi_2\gamma_2\psi_2$ ($d=9$) and $\varphi_2\gamma_2\psi_2$ ($d=10$). Let us determine $\F_2[\sigma^2,\varphi_2\psi_2,\varphi_2\gamma_2\psi_2]\otimes\Lambda_{\F_2}(\sigma\psi_2,\beta_2\varphi_2\psi_2,\beta_2\varphi_2\gamma_2\psi_2)$ as graded $\F_2$-vector space in degrees $\leq10$ (multiplication in polynomial and exterior algebras is denoted by $\cdot$).\bigskip

\begin{tabular}{|c|l|c|}
\hline
Degree &$\F_2$-basis for $H_*(K(\Z/2,2);\F_2)$ &$\dim_{\F_2}$\\
\hline
0 &1&1\\
1 &0&0\\
2 &$\sigma^2$ &1\\
3 &$\sigma\psi_2$ &1\\
4 &$\gamma_2\sigma^2$ &1\\
5 &$\sigma^2\otimes\sigma\psi_2$, $\beta_2\varphi_2\psi_2$ &2\\
6 &$\sigma^2\cdot\gamma_2\sigma^2$, $\varphi_2\psi_2$ &2\\
7 &$\gamma_2\sigma^2\otimes\sigma\psi_2$, $\sigma^2\otimes\beta_2\varphi_2\psi_2$ &2\\
8 &$\gamma_2^2\sigma^2$, $\sigma\psi_2\cdot\beta_2\varphi_2\psi_2$, $\sigma^2\cdot\varphi_2\psi_2$ &3\\
9 &$\gamma_2\sigma^2\otimes\beta_2\varphi_2\psi_2$, $\varphi_2\psi_2\otimes\sigma\psi_2$, $\beta_2\varphi_2\gamma_2\psi_2$ &3\\
10 &$\gamma_2\sigma^2\cdot\varphi_2\psi_2$, $\varphi_2\gamma_2\psi_2$ &2\\
\hline\end{tabular}
\end{exmp*}

Of course, this result agrees with table \ref{A:integral homology of K(Z/2,2)}, p. \pageref{A:integral homology of K(Z/2,2)}.

\endinput
%%%%%%%%%%%%%%%%%%%%%%%%%%%%%%%%%%
\chapter{Equivalence of J.-P. Serre and H. Cartan Descriptions}

\begin{Adefn}
Let $(-)^-:{\mathcal S}\to{\mathcal S}$ be the application given by $I^-=(\overset\vee a_0,a_1,\dots,a_k)$ for all $I=(a_0,\dots,a_k)\in{\mathcal S}$ ($I^-$ is clearly admissible).
\end{Adefn}

\begin{Alem}
Let $n\geq1$, $s\geq1$ and $I\in\Sigma^n{\mathcal S}$. Then
\begin{align}
\label{f:vee1}&e(I^-)\leq e(I),\\ 
\label{f:vee2}&I=(\deg(I^-),I^-)&&\text{if $e(I)=n$,}\\
\label{f:vee3}&\deg(I)=2\deg(I^-)\equiv0\ (2)&&\text{if $e(I)=n$ and}\\
\label{f:vee4}&Sq^I_s u_n=(Sq^{I^-}_s u_n)^2&&\text{if $e(I)=n$.}
\end{align}
\end{Alem}

\begin{proof}
To prove \eqref{f:vee1} it suffices to see that $e(I^-)=2a_1-\degst(I^-)=2a_1-(\degst(I)-a_0)=(a_0+2a_1)-\degst(I)\leq2a_0-\degst(I)=e(I)$ since $2a_1\leq a_0$ by admissibility of $I$. Suppose now that $e(I)=n$. To prove \eqref{f:vee2} it suffices to see that $\deg(I^-)=\deg(I)-a_0=\degst(I)+n-a_0=2a_0-e(I)+n-a_0=2a_0-n+n-a_0=a_0$. To prove \eqref{f:vee3} it suffices to see that $\deg(I)=\deg(\deg(I^-),I^-)=\deg(I^-)+\degst(I^-)+n=2\deg(I^-)$. Finally, to prove \eqref{f:vee4} it suffices to see that $Sq^I_s u_n=Sq^{\deg(I^-),I^-}_s u_n=Sq^{\deg(I^-)}Sq^{I^-}_s u_n=(Sq^{I^-}_s u_n)$ since $\deg(I^-)=\degst(I^-)+n=\deg(Sq^{I^-}_s u_n)$.
\end{proof}

\begin{Adefn}
The {\bf height} is an application $h:{\mathcal S}\to\N$ defined for all $I=(a_0,\dots,a_k)\in{\mathcal S}$ by
$$
h(I)=\begin{cases}
e(I)+1 &\text{if $a_0$ is even and}\\
e(I) &\text{if $a_0$ is odd.}
\end{cases}
$$
Let ${\mathcal S}_{q,n}$ be the graded subset given by all admissible sequences $I\in{\mathcal S}$ such that $\degst(I)=q$ and $h(I)\leq n$. Let ${\mathcal S}_{*,n}=\cup_{q\geq0}{\mathcal S}_{q,n}$.
\end{Adefn}

\begin{Alem}\label{l:l:Psi_S and Psi_C}
Let $n\geq1$ and $s\geq1$. We have the following graded set equality
$$
(\Sigma^n{\mathcal S}_{*,n})^-=(\Sigma^n\{I\in{\mathcal S}\ |\ e(I)<n\})^-.
$$ This induces the following homomorphism of graded $\F_2$-vector spaces
$$
\Psi_S:\Lambda_{\F_2}((\Sigma^n{\mathcal S}_{*,n})^-)\to\F_2[Sq^I_s u_n\ |\ \text{$I\in{\mathcal S}$ and $e(I)<n$}].
$$
The application
$$
\Psi_C:\F_2[Sq^I_s u_n\ |\ \text{$I\in{\mathcal S}$ and $e(I)<n$}]\to\Lambda_{\F_2}((\Sigma^n{\mathcal S}_{*,n})^-)
$$ defined by
$$
\prod_{1\leq i\leq k} Sq^{I_i}_s u_n\mapsto \prod_{1\leq i\leq k}\begin{cases}
0 &\text{if $\deg(I_i)\equiv0(2)$ and}\\
I_i &\text{if $\deg(I_i)\equiv1(2)$,}
\end{cases}
$$ is a morphism of graded $\F_2$-vector spaces for which $\Psi_S$ is a section, i.e. $\Psi_C\Psi_S=\id_{\Lambda_{\F_2}((\Sigma^n{\mathcal S}_{*,n})^-)}$. In particular, $\Psi_S$ is a monomorphism of graded $\F_2$-vector spaces.
\end{Alem}

\begin{proof}
Let $I\in(\Sigma^n{\mathcal S}_{*,n})^-$. Suppose that $e(I)=n$. Formula \eqref{f:vee3} implies that $\deg(I)$ is even, a contradiction. Hence $e(I)<n$ and $I\in(\Sigma^n\{I\in{\mathcal S}\ |\ e(I)<n\})^-$. The other inclusion is obvious. This proves the equality. The map $\Psi_C$ is defined on basis elements and therefore is linear. It is routine to check that $\Psi_S$ is a section.
\end{proof}

\begin{Adefn}
For all $n\geq0$, let $(-)^\square:\Sigma^n{\mathcal S}\to\Sigma^n{\mathcal S}$ be the application given by $I^\square=(\degst(I)+\max(n,e(I)),I)$ for all $I=(a_0,\dots,a_k)\in\Sigma^n{\mathcal S}$ ($I^\square$ is admissible since $\degst(I)+\max(n,e(I))-2a_0=\max(n,e(I))-e(I)\geq0$).
\end{Adefn}

\begin{Alem}
Let $n\geq1$ and $s\geq1$. Then
\begin{align}
\label{f:square1}&\deg(I^\square)\geq2\deg(I),\\
\label{f:square2}&I^\square=(\deg(I),I) &\text{if $e(I)\leq n$,}\\
\label{f:square3}&\deg(I^\square)=2\deg(I)\equiv0(2) &\text{if $e(I)\leq n$,}\\
\label{f:square4}&Sq^{I^\square}_s u_n=(Sq^I_s u_n)^2 &\text{if $e(I)\leq n$ and}\\
\label{f:square5}&h(I^\square)=\begin{cases}
n+1 &\text{if $\deg(I)\equiv0(2),$}\\
n &\text{if $\deg(I)\equiv1(2)$,}
\end{cases}&\text{if $e(I)\leq n$.}
\end{align}
\end{Alem}

\begin{proof}
To prove \eqref{f:square1} it suffices to see that $\deg(I^\square)=\degst(I^\square)+n=\degst(I)+\max(n,e(I))+\degst(I)+n\geq\degst(I)+n+\degst(I)+n=2(\degst(I)+n)=2\deg(I)$. Suppose now that $e(I)\leq n$. To prove \eqref{f:square2} it suffices to see that $I^\square=(\degst(I)+\max(n,e(I)),I)=(\degst(I)+n,I)=(\deg(I),I)$. To prove \eqref{f:square3} it suffices to see that $\deg(I^\square)=\degst(I^\square)+n=\degst(\deg(I),I)+n=\deg(I)+\degst(I)+n=\deg(I)+\deg(I)=2\deg(I)$. To prove \eqref{f:square4} it suffices to see that $Sq^{I^\square}_s u_n=Sq^{(\deg(I),I)}_s u_n=Sq^{\deg(I)}Sq^I_s u_n=(Sq^I_s u_n)$ since $\deg(I)=\degst(I)+n=\deg(Sq^I_s u_n)$. Finally, to prove \eqref{f:square5} it suffices to see that $e(I^\square)=2(\degst(I)+\max(n,e(I)))-\degst(I^\square)=\max(n,e(I))=n$ and $\degst(I)+\max(n,e(I))=\degst(I)+n=\deg(I)$. Therefore we have
\begin{align*}
h(I^\square)&=\begin{cases}
e(I^\square)+1&\text{if $\degst(I)+\max(n,e(I))\equiv0(2)$,}\\
e(I^\square)&\text{if $\degst(I)+\max(n,e(I))\equiv1(2)$,}
\end{cases}\\&=\begin{cases}
n+1&\text{if $\deg(I)\equiv0(2)$,}\\
n&\text{if $\deg(I)\equiv0(2)$.}
\end{cases}
\end{align*}
\end{proof}

\begin{Anota}
We will write $(\Sigma^n{\mathcal S}_{*,n})^\square$ the graded set $\{I\in(\Sigma^n{\mathcal S}_{*,n})^+\ |\ e(I)=n\}$.
\end{Anota}

\begin{Alem}\label{l:deg(I) cong 2 (4)}
If $I\in(\Sigma^n{\mathcal S}_{*,n})^\square$ then $\deg(I)\equiv2(4)$.
\end{Alem}

\begin{proof}
Following \eqref{f:vee2} and \eqref{f:vee3} we have $I=(\deg(I^-),I^-)$ and $\deg(I)=2\deg(I^-)$. If $\deg(I^-)\equiv0(2)$ then $h(I)=e(I)+1=n+1>n$, a contradiction. Thus $\deg(I^-)\equiv1(2)$ and therefore $\deg(I)\equiv2(4)$.
\end{proof}

\begin{Alem}
The applications $(-)^-$ and $(-)^\square$ induce the bijection
$$
(-)^-:(\Sigma^n{\mathcal S}_{*,n})^\square\longleftrightarrow(\Sigma^n{\mathcal S}_{*,n})^-:(-)^\square
$$
\end{Alem}

\begin{proof}
By \eqref{f:vee2} we have$(I^-)^\square=(\deg(I^-),I^-)=I$ and by \eqref{f:square2} we have $(I^\square)^-=(\deg(I),I)^-=I$.
\end{proof}

\begin{Alem}\label{l:Phi_S and Phi_C}
Let $n\geq1$ and $s\geq1$. Let 
$$
\Phi_S:\F_2[(\Sigma^n{\mathcal S}_{*,n})^+]\to\F_2[Sq^I_s u_n\ |\ \text{$I\in{\mathcal S}$ and $e(I)<n$}]
$$ be the morphism of graded $\F_2$-algebras defined by
$$
I\mapsto Sq^I_s u_n.
$$ The application
$$
\Phi_C:\F_2[Sq^I_s u_n\ |\ \text{$I\in{\mathcal S}$ and $e(I)<n$}]\to\F_2[(\Sigma^n{\mathcal S}_{*,n})^+]
$$ defined by
$$
\prod_{1\leq i\leq k}(Sq^{I_i}_s u_n)^{r_i}\mapsto\prod_{1\leq i\leq k}\begin{cases}
I_i^{r_i} &\text{if $\deg(I_i)\equiv0(2)$,}\\
(I_i^\square)^{r_i\over 2} &\text{if $\deg(I_i)\equiv1(2)$ and $r_i\equiv0(2)$,}\\
0 &\text{if $\deg(I_i)\equiv1(2)$ and $r_i\equiv1(2)$.}
\end{cases}
$$ when all the $I_i$ are distincts (i.e. $\#\{I_i\ |\ 1\leq i\leq k\}=k$), is a morphism of graded $\F_2$-vector spaces for which $\Phi_S$ is a section, i.e. $\Phi_C\Phi_S=\id_{\F_2[(\Sigma^n{\mathcal S}_{*,n})^+]}$. In particular, $\Phi_S$ is a monomorphism of graded $\F_2$-vector spaces.
\end{Alem}

\begin{proof}
The map $\Phi_C$ is defined on basis elements and therefore is linear. Let us check that $\Phi_S$ is a section. Let $\prod_i I_i^{r_i}\in\F_2[(\Sigma^n{\mathcal S}_{*,n})^+]$ with all the $I_i$ distincts. We have
\begin{align*}
\Phi_S(\prod_i I_i^{r_i}) &=\prod_i(Sq^{I_i}_s u_n)^{r_i}\\
&= \prod_{e(I_i)=n}(Sq^{I_i}_s u_n)^{r_i}\prod_{e(I_i)<n}(Sq^{I_i}_s u_n)^{r_i}\\
&= \prod_{e(I_i)=n}(Sq^{I_i^-}_s u_n)^{2r_i}\prod_{e(I_i)<n}(Sq^{I_i}_s u_n)^{r_i} &\text{by \eqref{f:vee4}.}
\end{align*}
Let us remark that $I_i^-\not=I_j$ for all $i$ and $j$. Suppose the converse. Then there are $i$ and $j$ such that $I_i^-=I_j$. Thus $I_i^-\equiv0(2)$ and therefore $I_i\equiv0(4)$, which contradicts Lemma \ref{l:deg(I) cong 2 (4)}. It suffices then to show that $\Phi_C((Sq^{I_i^-}_s u_n)^{2r_i})=((I_i^-)^\square)^{2r_i\over2}=I_i^{r_i}$.
\end{proof}

\begin{Alem}
We have the relations
$$\Phi_C\Psi_S=0=\Psi_C\Phi_S$$
with $\Psi_S$, $\Psi_C$ as in Lemma \ref{l:l:Psi_S and Psi_C} and $\Phi_S$, $\Phi_C$ as in Lemma \ref{l:Phi_S and Phi_C}.
\end{Alem}

\begin{proof}
The map $\Phi_C$ is zero on elements of odd degree and thus $\Phi_C\Psi_S=0$.
The map $\Psi_C$ is zero on elements of even degree and thus $\Psi_C\Phi_S=0$.\end{proof}

%%%

\begin{Alem}
We have the following Poincar\'e serie
$$
P(\F_2[Sq^I u_n\ |\ \text{$I\in{\mathcal S}$ and $e(I)<n$}],t)=\prod_{\substack{I\in{\mathcal S}\\ e(I)<n}}\frac{1}{1-t^{\deg(I)}}
$$
\end{Alem}

\begin{proof}
On v�rifie faciAlement que pour toute suite admissible $I$ telle que $e(I)<n$ on a
$$P(\F_2[Sq^I u_n],t)=1+\sum_{r\geq0}t^{2^r\deg(I)}=\frac{1}{1-t^{\deg(I)}}.$$
On conclut en remarquant que l'isomorphisme d'alg�bres
$$
\F_2[Sq^I u_n\ |\ \text{$I\in{\mathcal S}$ et $e(I)<n$}]\cong
\bigotimes_{\substack{I\in{\mathcal S}\\ e(I)<n}} \F_2[Sq^I u_n]
$$
implique que
$$
P(\F_2[Sq^I u_n\ |\ \text{$I\in{\mathcal S}$ et $e(I)<n$}],t)=
\prod_{\substack{I\in{\mathcal S}\\ e(I)<n}}P(\F_2[Sq^I u_n],t)
$$
\end{proof}

\begin{Alem}
On a la s�rie de Poincar�
$$
P(\Lambda_{\F_2}[(\Sigma^n{\mathcal S}_{*,n})^-],t)=\prod_{I\in(\Sigma^n{\mathcal S}_{*,n})^-}1+t^{\deg(I)}
$$
\end{Alem}

\begin{Athm}\label{t:equiv_SEp_CAp}
On a un isomorphisme de $\F_2$-espaces vectoriels gradu�s
$$
SE':\Lambda_{\F_2}((\Sigma^n{\mathcal S}_{*,n})^-)\otimes_{\F_2} \F_2[(\Sigma^n{\mathcal S}_{*,n})^+]\cong
$$
$$
\F_2[Sq^I u_n\ |\ \text{$I\in{\mathcal S}$ et $e(I)<n$}]:CA'
$$
L'isomorphisme $SE'$ est donn� par
$$
SE':{\mathcal I}\otimes{\mathcal J}\mapsto\Psi_S({\mathcal I})\Phi_S({\mathcal J})
$$ et l'isomorphisme $CA'$ est donn� par
$$
CA':\prod_{1\leq i\leq k}Sq^{I_j} u_n\mapsto\prod_{1\leq i\leq k}(\Psi_C\otimes1+1\otimes\Phi_C)(Sq^{I_j}u_n)
$$ avec $\Psi_S$, $\Psi_C$ comme dans le Lemme \ref{l:mono_ev} et $\Phi_S$, $\Phi_C$ comme dans le Lemme \ref{l:mono_alg}.
\end{Athm}

\begin{proof}
Remarquons tout d'abord que les deux $\F_2$-espaces vectoriels ci-dessus ont les m�mes s�ries de Poincar� puisque
\begin{flalign*}
&\frac{P(\F_2[Sq^I u_n\ |\ \text{$I\in{\mathcal S}$ et $e(I)<n$}],t)}{P(\F_2[(\Sigma^n{\mathcal S}_{*,n})^+],t)}
=\frac{\displaystyle\prod_{\substack{I\in{\mathcal S}\\ e(I)<n}}\frac{1}{1-t^{\deg(I)}}}{\displaystyle\prod_{I\in(\Sigma^n{\mathcal S}_{*,n})^+}\frac{1}{1-t^{\deg(I)}}}=\\
&\frac{\displaystyle\prod_{I\in(\Sigma^n{\mathcal S}_{*,n})^+}1-t^{\deg(I)}}{\displaystyle\prod_{\substack{I\in{\mathcal S}\\ e(I)<n}}1-t^{\deg(I)}}
\qquad=\qquad\frac{\displaystyle\prod_{I\in(\Sigma^n{\mathcal S}_{*,n})^\square}1-t^{\deg(I)}}{\displaystyle\prod_{\substack{I\in{\mathcal S}\\ e(I)<n\\ \text{$\deg(I)$ impair}}}1-t^{\deg(I)}}
\end{flalign*}\begin{alignat*}{2}
=\qquad&\frac{\displaystyle\prod_{I\in(\Sigma^n{\mathcal S}_{*,n})^\square}1-t^{\deg(I)}}{\displaystyle\prod_{I\in(\Sigma^n{\mathcal S}_{*,n})^-}1-t^{\deg(I)}}&&\qquad\text{par le Lemme \ref{l:mono_ev}}\\
=\qquad&\prod_{I\in(\Sigma^n{\mathcal S}_{*,n})^-}\frac{1-t^{\deg(I^\square)}}{1-t^{\deg(I)}}&&\qquad\text{par le Lemme \ref{l:moins-carre}}\\
=\qquad&\prod_{I\in(\Sigma^n{\mathcal S}_{*,n})^-}\frac{1-t^{2\deg(I)}}{1-t^{\deg(I)}}&&\qquad\text{d'apr�s la definition \ref{d:square}}\\
=\qquad&\prod_{I\in(\Sigma^n{\mathcal S}_{*,n})^-}1+t^{\deg(I)}\\=\qquad& P(\Lambda_{\F_2}((\Sigma^n{\mathcal S}_{*,n})^-),t).
\end{alignat*}
Pour conclure il suffit par exemple de voir que $SE'$ est injective. Le Lemme \ref{l:relations_SE_CA} assure que $SE'$ est une section de $CA'$, d'o� le r�sultat.
\end{proof}

\begin{Adefn}
On d�finit une application $S:{\mathcal W}\to{\mathcal S}$ en proc�dant comme suit. Pour tout mot admissible on choisit un mot brut $\alpha$ qui le repr�sente et qui ne contient pas la lettre $\beta_2$. On num�rote � partir de $0$ et de gauche � droite les lettres de $\alpha$ qui sont $\gamma_2$, $\varphi_2$ ou $\psi_{2^f}$ et on d�signe par $k$ le nombre de ces lettres. On note $\alpha_i$ la $i$-�me de ces lettres et $\bar\alpha_i$ le mot brut form� des lettres de $\alpha$ qui suivent la $i$-�me de ces lettres. On pose $a_i=\deg(\bar\alpha_i)+h(\alpha_i)$ pour tout $0\leq i\leq k-1$ et $a_i=0$ pour tout $i\geq k$. On pose $S(\alpha)=(a_i)_{i\geq0}$.
\end{Adefn}

\begin{Alem}
L'application $S:{\mathcal W}\to{\mathcal S}$ pr�serve les degr�s stables.
\end{Alem}

\begin{proof}
\end{proof}

\begin{Adefn}
Pour tout entier $k\geq0$ on d�finit une application $W^k:{\mathcal S}\to{\mathcal W}$ en proc�dant comme suit. Pour toute suite admissible $I=(a_i)_{i\geq0}$ on consid�re les lettres $\alpha_i$ d�finies par
$$
\alpha_i=\begin{cases}
() &\text{si $a_i=0$, }\\
\gamma_2 &\text{si $a_i\not=0$ est pair et}\\
\varphi_2 &\text{si $a_i$ est impair.}
\end{cases}
$$
On consid�re �gaAlement la suite $(l_i)_{i\geq0}$ d�finie par
$$
l_i=a_{i}-h(\alpha_{i})-2a_{i+1}
$$
On pose $W^k(I)=\sigma^{k-h(\alpha_0)+1}\alpha_0\sigma^{l_0}\alpha_1\sigma^{l_1}\dots$.
\end{Adefn}

\begin{Alem}
Les applications $W^k:{\mathcal S}\to{\mathcal W}$ pr�servent les degr�s stables pour tout $k\geq0$.
\end{Alem}

\begin{proof}
\end{proof}

\begin{Athm}\label{t:equiv_S_W}
Les applications $S$ et $W^n$ induisent la bijection d'ensembles gradu�s
$$
S^n:{\mathcal W}_{q,n}\longleftrightarrow\Sigma^n{\mathcal S}_{q,n}:W^n
$$ pour tout $n\geq1$ et $q\geq0$. En particulier, pour tout $n\geq1$, on a la bijection d'ensembles gradu�s
$$
S^n:{\mathcal W}_{*,n}\longleftrightarrow\Sigma^n{\mathcal S}_{*,n}:W^n
$$ qui induit l'isomorphisme d'alg�bres gradu�es
$$
S^n:\Lambda_{\F_2}({\mathcal W}_{*,n}^-)\otimes_{\F_2} \F_2[{\mathcal W}_{*,n}^+]\cong
$$
$$
\Lambda_{\F_2}((\Sigma^n{\mathcal S}_{*,n})^-)\otimes_{\F_2} \F_2[(\Sigma^n{\mathcal S}_{*,n})^+]:W^n
$$
\end{Athm}

\begin{proof}

\end{proof}

\begin{Athm}\label{t:pont_mod_2}
On a un isomorphisme de $\F_2$-espaces vectoriels gradu�s
$$
SE:\Lambda_{\F_2}({\mathcal W}_{*,n}^-)\otimes_{\F_2} \F_2[{\mathcal W}_{*,n}^+]\cong
$$
$$
\F_2[Sq^I u_n\ |\ \text{$I\in{\mathcal S}$ et $e(I)<n$}]:CA
$$
L'isomorphisme $SE$ est donn� par
$$
SE=SE'\circ S
$$ et l'isomorphisme $CA$ est donn� par
$$
CA=W^n\circ CA'
$$ avec $S^n$, $W^n$ comme dans le th�or�me \ref{t:equiv_S_W} et $SE'$, $CA'$ comme dans le th�or�me \ref{t:equiv_SEp_CAp}. Cet isomorphisme munit $\Lambda_{\F_2}({\mathcal W}_{*,n}^-)\otimes_{\F_2} \F_2[{\mathcal W}_{*,n}^+]$ d'une structure de ${\mathcal A}_2$-module induite par la description de J.-P. Serre de $H^*(K(\Z/2,n);\F_2)$.
\end{Athm}

\endinput