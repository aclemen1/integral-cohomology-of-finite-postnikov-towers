\section{Organization of the work}

Chapter 2 is devoted to the study of the (co)homology of Eilenberg-MacLane spaces. The main results of J.-P. Serre \cite{Se53} and H. Cartan \cite{Ca55} are exposed and developped. The particular aspects of duality between mod-$2$ homology and cohomology are studied and a proof of Theorem \ref{thm_main_technical} is given.

\medskip
Chapter 3 presents the ``Eilenberg-MacLane machine'' which is a C++ program designed to compute explicitely integral homology groups of Eilenberg-MacLane spaces. The main algorithms are quoted and some heuristic results derived from the computations of the machine are proved, namely Propositions \ref{prop_heuristic_H} and \ref{prop_heuristic_coH}.

\medskip
Chapter 4 provides examples of spaces which retract, respectively do not retract, onto an Eilenberg-MacLane space, and thus classify them. A general classification result for a wide family of H-spaces is given. One can find the proofs of Theorems \ref{thm_retract}, \ref{thm_slight_gen} and \ref{thm_no_retract} in this chapter.

\medskip
Chapter 5 exploits transverse implications provided by Eilenberg-MacLane spaces to detect transverse elements in some $2$-stage Postnikov systems. The two main results of this work, Theorems \ref{thm_first_main} and \ref{thm_second_main}, as well as the requisite technical stuff to prove them, are established in the last sections.

\endinput