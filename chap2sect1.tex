\section{J.-P. Serre's description}\label{s:Serre's description}

\begin{defn}\index{admissible sequence}\index{stable degree}\index{excess}\index{0@$\mathcal S$}
A non-empty finite sequence of positive integers $I=(a_0,\dots,a_k)$, where $k$ is varying, is {\bf admissible} if $a_i\geq2a_{i+1}$ for all $0\leq i\leq k-1$. Let $\mathcal S$ be the set of all such admissible sequences. The {\bf stable degree} is a map $\degst:{\mathcal S}\to\N$ defined by $\degst(I)=\sum_{i=0}^k a_i$ for all $I=(a_0,\dots,a_k)\in{\mathcal S}$. The stable degree induces a {\bf grading} on the set $\mathcal S$ of all admissible sequences. The {\bf excess} is a map $e:{\mathcal S}\to\N$ defined by $e(I)=2a_0-\degst(I)=a_0-\sum_{i=1}^k a_i$ for all $I=(a_0,\dots,a_k)\in{\mathcal S}$.
\end{defn}

\begin{conv}\label{convention on squares}
Let $n\geq1$ and $s\geq1$. Consider the fundamental class $\iota_n\in H^n(K(\Z/2^s,n);\Z/2^s)$ and its mod-$2$ reduction $u_n\in H^n(K(\Z/2^s,n);\F_2)$. Let $I=(a_0,\dots,a_k)$ be an admissible sequence. We will write $Sq^I_s u_n$ instead of $Sq^{a_0}\dots Sq^{a_{k-1}}\delta_s\iota_n$ (usually denoted by $Sq^{a_0,\dots,a_{k-1}}\delta_s\iota_n$) if $a_k=1$ and instead of $Sq^{a_0}\dots Sq^{a_k} u_n$ (also denoted by $Sq^{a_0,\dots,a_k} u_n$ or $Sq^I u_n$) if $a_k\not=1$. Here $\delta_s$ denotes the connecting homomorphism associated to $\xymatrix{0\ar[r] &\Z/2^s\ar[r] &\Z/2^{s+1}\ar[r] &\Z/2\ar[r] &0}$. In particular, since $\delta_1=Sq^1$ and the reduction is the identity when $s=1$, we have $Sq^I_1 u_n=Sq^I u_n$.
\end{conv}

Let $R$ be a ring and $X$ be a space. For any $n\geq0$, the cohomology group $H^n(X;R)$ is a left $R$-module. Therefore, the set $\{H^n(X;R)\ |\ n\geq0\}$ of all cohomology groups is a graded $R$-module which we denote $H^*(X;R)$. Moreover, $H^*(X;R)$, endowed with the cup product, is a graded $R$-algebra.

\medskip
In 1953, J.-P. Serre computed the mod-$2$ cohomology of Eilenberg-MacLane spaces and stated the following result:

\begin{thm}\label{t:Serre's description}
Let $n\geq1$ and $s\geq1$. The graded $\F_2$-algebra $H^*(K(\Z/2^s,n);\F_2)$ is isomorphic to the graded polynomial $\F_2$-algebra on generators $Sq^I_s u_n$, where $I$ covers all the admissible sequences of excess $e(I)<n$ and $u_n$ is the reduction of the fundamental class (see convention \ref{convention on squares}). The degree of a generator $Sq^I_s u_n$ is $\deg(Sq^I_s u_n)=\degst(I)+n$.
\end{thm}

\begin{proof}
See \cite[Th\'eor\`eme 2, p. 203 and Th\'eor\`eme 4, p. 206]{Se53}.
\end{proof}

This result explicits the $\A_2$-module structure of $H^*(K(\Z/2^s,n);\F_2)$. In particular, this is a differential graded $\F_2$-algebra under the Bockstein homomorphism $\beta=Sq^1$. 

For instance, let us determine the unstable $\A_2$-module structure of the algebra $H^*(K(\Z/2,1);\F_2)$. The excess of an admissible sequence $I$ is zero if and only if $I=(0)$ and the fundamental class $u_1\in H^1(K(\Z/2,1);\F_2)$ is then the only generator. Therefore we have
$$
H^*(K(\Z/2,1);\F_2)\cong \F_2[u_1]
$$ and the $\A_2$-action is given pictorially as follows
$$\xymatrix@R=-0.1cm{
\bullet &\bullet\ar@/^10pt/[r]^-{Sq^1} &\bullet\ar@/^20pt/[rr]^-{Sq^2} &\bullet\ar@/^10pt/[r]_-{Sq^1} &\bullet &\bullet\ar@/^10pt/[r]^-{Sq^1} &\bullet \\
1 &u_1 &u_1^2 &u_1^3\ar@/_10pt/[rr]_-{Sq^2}\ar@/_30pt/[rrr]_-{Sq^3} &u_1^4 &u_1^5 &u_1^6 &\dots
}$$

In particular, this example shows the Adem relation $Sq^1Sq^2=Sq^3$. Have a glance at Appendix \ref{a:Adem relations} for more Adem relations.

\bigskip
It is well known that an Eilenberg-MacLane space associated with an abelian group has a unique H-space structure up to homotopy. The H-space structure can be seen as inherited from the loop space structure or from the addition law of the associated abelian group. Therefore, the graded $\F_2$-algebra $H^*(K(\Z/2^s,n);\F_2)$ is also a differential graded Hopf algebra. 

\begin{thm}\label{t:Hopf structure of K(G,n) cohomology}
Let $n\geq1$ and $s\geq1$. The differential graded $\F_2$-algebra $H^*(K(\Z/2^s,n);\F_2)$ is a connected, biassociative, bicommutative and primitively generated differential graded Hopf algebra.
\end{thm}

\begin{proof}
See \cite[pp. 54-55]{Sm70}.
\end{proof}

It is now easy to determine the modules of primitives and indecomposables of $H^*=H^*(K(\Z/2^s,n);\F_2)$. The modules of indecomposable elements is clearly given by
\begin{align*}
QH^*&\cong\F_2\{Sq^I_s u_n\ |\ \text{$I$ admissible and $e(I)<n$}\},\\
&\text{the graded $\F_2$-vector space generated by}\\
&\text{all the $Sq^I_s u_n$ with $I$ admissible and $e(I)<n$.}\\
\end{align*}
Since $H^*$ is primitively generated, the Milnor-Moore theorem (see p. \pageref{t:Milnor-Moore}) gives the following short exact sequence of graded $\F_2$-vector spaces:
$$\xymatrix{
0\ar[r] &P(\xi H)\ar[r] &PH\ar[r] &QH\ar[r] &0.
}$$ Therefore, every indecomposable element is primitive and every primitive element which is decomposable is a square of a primitive element. Thus we have
\begin{align*}
PH^*&\cong\F_2\{(Sq^I_s u_n)^{2^i}\ |\ \text{$I$ admissible, $e(I)<n$ and $i\geq0$}\},\\
&\text{the graded $\F_2$-vector space generated by}\\
&\text{all the iterated squares of $Sq^I_s u_n$ with $I$ admissible and $e(I)<n$.}\\
\end{align*}


\begin{defn}\index{Poincar\'e series}
Let $H^*$ denote a (positively) graded vector space over a field $k$. The {\bf Poincar\'e series} of $H^*$ is the formal power series $$P(H^*,t)=\sum_{i\geq0}\dim_k H^i\cdot t^i\in\Z[[t]].$$
\end{defn}

\begin{prop}\label{p:Poincare serie for K(Z/2,n)}
Let $P_n(t)$ denote the Poincar\'e series $P(H^*(K(\Z/2^s,n);\F_2),t)$. We have
$$
P_n(t)=\prod_{h_1\geq\dots\geq h_{n-1}\geq0}{1\over1-t^{1+2^{h_1}+\dots+2^{h_{n-1}}}}.
$$
\end{prop}

\begin{proof}
See \cite[Th\'eor\`eme 1 and formula (17.7), pp. 211-212]{Se53} or Lemma \ref{l:a proof for Serre's result}, p. \pageref{l:a proof for Serre's result}.
\end{proof}

\begin{defn}\index{Serre function}
The {\bf Serre function} associated to $P_n(t)$ is the continuous map given by
$$
\varphi_n(x)=\log_2 P_n(1-2^{-x})
$$ for all $x\in[0,\infty[$.
\end{defn}

\begin{thm}\label{t:2.3-Serre}
We have the asymptotic growth formula
$$
\varphi_n(x)\sim {x^n\over n!}\hskip0.8truecm\text{i.e.}\hskip0.8truecm
\lim_{x\to+\infty}{\varphi_n(x)\over {x^n\over n!}}=1.
$$
\end{thm}

\begin{proof}
See \cite[Th\'eor\`eme 6, pp. 215-216]{Se53}.
\end{proof}