\section{Proof of the second main theorem}\label{s:proof of thm on 2-stage Postnikov systems}
%\section{Proof of Theorem \ref{t:no exponent for some Postnikov pieces}}

\begin{lem}\label{l:sub-Hopf of primitive}
Let $H$ be a primitive Hopf algebra. If $H'$ is a sub-Hopf algebra, then $H'$ is primitive.
\end{lem}

\begin{proof}
For a direct proof, see \cite[Proposition 2.6, p. 69]{Sm70}. For a better result, see \cite[Proposition 6.3, p. 247]{MM65}.
\end{proof}

\begin{lem}\label{l:indecomposables}
Let $H$ be a cocommutative primitive Hopf algebra, $H'$ be a polynomial Hopf algebra and $\varphi:H\to H'$ be a Hopf algebra map. Then $Q\subker \varphi=\subker\varphi\cap QH$.
\end{lem}

\begin{proof}
If $x\not=0\in\subker\varphi$ is indecomposable, then it is primitive in $H$ since $\subker\varphi$ is a primitively generated sub-Hopf algebra by Lemma \ref{l:sub-Hopf of primitive}. If it is decomposable in $H$, then it is an iterated square of a primitive by Milnor-Moore theorem, say $x=w^{2^n}$, and $\varphi(x)=\varphi(w^{2^n})=\varphi(w)^{2^n}=0$. This implies that $w\in\subker\varphi$ since $H'$ is polynomial. Therefore, $x$ is decomposable in $\subker\varphi$, which is a contradiction. Thus $x$ is indecomposable in $H$. The other inclusion is obvious.
\end{proof}

\begin{lem}\label{l:kernel}
Let $1\leq m\leq n$, $KG$ be a strictly $m$-anticonnected GEM, $KH$ be a strictly $n$-anticonnected GEM and $f:KG\to K\Sigma H$ a continuous map. Consider the induced map of graded vector spaces $f^*:H^*(K\Sigma H;\F_2)\to H^*(KG;\F_2)$. For all $0<\delta\leq n-m$ we have
$$
\subker f^*\cap QH^*_{[\delta]}(K\Sigma H)\not=0.
$$
\end{lem}

\begin{proof}
Suppose that $\ker f^*\cap H^*_{[\delta]}(K\Sigma H)=0$. Then we have
\begin{align*}
P_{[\delta]}(K\Sigma H)(t)\leq P_{[0]}(KG)(t)
\end{align*}
and thus by Lemma \ref{l:Serre improved}
\begin{align*}
&\rk(H_n)\cdot\frac{x^{n+1-\delta}}{(n+1-\delta)!}\sim\varphi_{[\delta]}(K\Sigma H)(x)\\
&\qquad\leq\varphi_{[0]}(KG)(x)\sim\rk(G_m)\cdot\frac{x^m}{m!}.
\end{align*} 
This contradicts the fact that $\delta\leq n-m$. Therefore 
$$\ker f^*\cap H^*_{[\delta]}(K\Sigma H)\not=0.$$
Let $x\not=0\in\ker f^*\cap H^*_{[\delta]}(K\Sigma H)$. Then $x=a\cdot k$ with $a\in H^*_{[\delta]}(K\Sigma H)$ and $k\not=0\in\overline{\subker f^*}\cap H^*_{[\delta]}(K\Sigma H)$. Since $\subker f^*$ is polynomial by Borel's results, it is generated (as an algebra) by its indecomposables and therefore $Q\subker f^*\cap H^*_{[\delta]}(K\Sigma H)\not=0$. Thus by Lemma \ref{l:indecomposables} we have $Q\subker f^*\cap H^*_{[\delta]}(K\Sigma H)=\subker f^*\cap QH^*_{[\delta]}(K\Sigma H)\not=0$.
\end{proof}

\vskip5cm
Our second main result is:

\begin{thm_second_main}
\input thm_second_main
\end{thm_second_main}

\begin{proof} %[Proof of Theorem \ref{t:no exponent for some Postnikov pieces}]
By hypothesis, $KG$ is strictly $m'$-anticonnected, with $m'\leq m$. By Lemma \ref{l:kernel} we have $\subker f^*\cap QH^*_{[2]}(K\Sigma H)\not=0$ since $n\geq m+2\geq m'+2$. Let $\sum_{1\leq i\leq j}Sq^{I_i}_{s_i}u_i\not=0\in\subker f^*\cap QH^*_{[2]}(K\Sigma H)$ with all the $Sq^{I_i}_{s_i}u_i$'s distinct. We have $e(I_i)<\deg(u_i)-2$ for all $1\leq i\leq j$. Renumber the $I_i$'s such that $I_1$ can be written $I_1=(a_0,\dots,a_k)$ with $a_0$ maximal. Let $d=\deg(\sum_{1\leq i\leq j}Sq^{I_i}_{s_i}u_i)$ and set
\begin{align*}
I'&=\begin{cases}
(2a_0) &\text{if $d\equiv1(2)$,}\\
(4a_0+2,2a_0+1) &\text{if $d\equiv0(2)$,}
\end{cases}\\
I''&=\begin{cases}
(2a_0+1) &\text{if $d\equiv1(2)$,}\\
(4a_0+3,2a_0+1) &\text{if $d\equiv0(2)$.}
\end{cases}
\end{align*} 
Let us remark that the sequences $(I',I_i)$ and $(I'',I_i)$ are all distinct admissible sequences since $a_0$ is maximal. Consider now $x=Sq^{I'}\sum_{1\leq i\leq j}Sq^{I_i}_{s_i}u_i=\sum_{1\leq i\leq j}Sq^{I',I_i}_{s_i}u_i$. One readily check that $x\in P\odd H^*(K\Sigma H;\F_2)=Q\odd H^*(K\Sigma H;\F_2)$. We have
\begin{align*}
\sigma^*Sq^1x &=\sigma^*Sq^1\sum_{1\leq i\leq j}Sq^{I',I_i}_{s_i}u_i\\
&=\sigma^*\sum_{1\leq i\leq j}Sq^1Sq^{I',I_i}_{s_i}u_i\\
&=\sigma^*\sum_{1\leq i\leq j}Sq^{I'',I_i}_{s_i}u_i.
\end{align*}
Renumber the $I_i$'s again such that $e(I'',I_i)<\deg(u_i)$ for all $1\leq i\leq j_0$ and $e(I'',I_i)\geq\deg(u_i)$ for all $j_0+1\leq i\leq j$. We clearly have $j_0\geq1$ since $e(I'',I_1)\leq e(I_1)+2<\deg(u_i)$. Then the equality becomes
\begin{align*}
\sigma^*Sq^1x&=\sigma^*\sum_{1\leq i\leq j_0}\underbrace{Sq^{I'',I_i}_{s_i}}_{e<\deg(u_i)}u_i+\sigma^*\sum_{j_0+1\leq i\leq j}\underbrace{Sq^{I'',I_i}_{s_i}}_{e\geq\deg(u_i)}u_i\\
&=\sigma^*\sum_{1\leq i\leq j_0}\underbrace{Sq^{I'',I_i}_{s_i}}_{e<\deg(u_i)}u_i+\sigma^*\sum\text{squares}\\
&=\sigma^*\sum_{1\leq i\leq j_0}\underbrace{Sq^{I'',I_i}_{s_i}}_{e<\deg(u_i)}u_i
\end{align*}
since $\sigma^*$ vanishes on decomposables. Consequently $\sigma^*Sq^1x\not=0$ because all the $Sq^{I'',I_i}_{s_i}u_i$ are distinct indecomposables and $\sigma^*:QH^*(GEM;\F_2)\to PH^*(\Omega GEM;\F_2)$ is in general a monomorphism (see \cite[Expos\'e 15, Proposition 3, p. 8]{Ca55}). Thus we have proved that $Sq^1x\not\in\ker\sigma^*$. We conclude the proof by using results of Section \ref{section:strategy}.
\end{proof}

\label{s:proof of thm on 2-stage Postnikov systems:discussion}
Let us make some closing comments on the methods that were used here. 

The hypothesis of Theorem \ref{t:no exponent for some Postnikov pieces} force $KH$ to be at least $(m+2)$-anticonnected when $KG$ is $m$-anticonnected. Suppose it is not the case and, for instance, $KH$ is strictly $(m+1)$-anticonnected as $KG$ is strictly $m$-anticonnected. Let us now reconsider the proof of the theorem. By Lemma \ref{l:kernel} we have $\subker f^*\cap QH^*_{[1]}(K\Sigma H)\not=0$. Thus if $Sq^I_s u\not=0\in\subker f^*\cap QH^*_{[1]}(K\Sigma H)$, then $e(I)<\deg(u)-1$ and $e(I'',I)\leq\deg(u)$. Suppose that $e(I'',I)=\deg(u)$. Then $\sigma^*Sq^{I'',I}_s u=0$ since $Sq^{I'',I}_s u$ is a square. In these conditions, it is not possible to ensure the existence of an element $x\in Q\odd H^*(K\Sigma H;\F_2)$ such that $Sq^1x\not\in\ker\sigma^*$ and the strategy developped in Section \ref{section:strategy} cannot be utilized.

It is possible to show that if $X[m]$ is a strictly $m$-anticonnected $2$-local Postnikov piece of finite type, then there is a generalized Serre function verifying the following asymptotic property:
$$
\varphi(X)(x)\stackrel{\leq}{\sim}\rk(\pi_mX)\cdot\frac{x^m}{m!}.
$$
Therefore, one might want to replace $KG$ by $X[m]$ in the Theorem and ask if it is still true. An adaptation of Lemma \ref{l:kernel} to this situation gives
$$
Q\subker f^*\cap H^*_{[\delta]}(K\Sigma H)\not=0.
$$ But Lemma \ref{l:indecomposables} is not valid anymore since $H^*(X[m];\F_2)$ is not polynomial in general. Thus we can only conclude that 
$$\subker f^*\cap PH^*_{[\delta]}(K\Sigma H)\not=0.$$ Suppose that $\subker f^*\cap PH^*_{[\delta]}(K\Sigma H)$ contains only squares. Then the method for proving Theorem \ref{t:no exponent for some Postnikov pieces} is compromised since cohomology suspension vanishes on decomposables.

\endinput
%%%%%%%%%%%%%%%%%%%%%%%%%%%%%%%%%

\begin{lem}
Let $G$ be a finitely generated $2$-torsion abelian group. For all $n>q\geq1$, we have a monomorphism 
$$\sigma^*:QH^*_{(q)}(G,n)\to QH^*_{(q)}(G,n-1).$$
\end{lem}

\begin{proof}
It is well known that $\sigma^*:QH^*(K(G,n);\F_2)\to PH^*(K(G,n);\F_2)$ is a monomorphism. Therefore the restriction to $QH^*_{(q)}(G,n)$ is also a monomorphism. Moreover, the suspension of an indecomposable in $QH^*_{(q)}(G,n)$ is not a square since its excess is lower than $n-1$ by hypothesis. Thus the Milnor-Moore theorem insures that it is an indecomposable.
\end{proof}