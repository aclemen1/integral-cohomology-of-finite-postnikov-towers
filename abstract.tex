\chapter*{Abstract}

By the work of H. Cartan, it is well known that one can find elements of arbitrarilly high torsion in the integral (co)homology groups of an Eilenberg-MacLane space $K(G,n)$, where $G$ is a non-trivial abelian group and $n\geq 2$.

\medskip
The main goal of this work is to extend this result to H-spaces having more than one non-trivial homotopy group.

\medskip
In order to have an accurate hold on H. Cartan's result, we start by studying the duality between homology and cohomology of $2$-local Eilenberg-MacLane spaces of finite type. This leads us to some improvements of H. Cartan's methods in this particular case.

\medskip
Our main result can be stated as follows. Let $X$ be an H-space with two non-trivial finite $2$-torsion homotopy groups. Then $X$ does not admit any exponent for its reduced integral graded (co)homology group.

\medskip
We construct a wide class of examples for which this result is a simple consequence of a topological feature, namely the existence of a weak retract $X\to K(G,n)$ for some abelian group $G$ and $n\geq2$.

\medskip
We also generalize our main result to more complicated stable \mbox{$2$-stage} Postnikov systems, using the Eilenberg-Moore spectral sequence and analytic methods involving Betti numbers and their asymptotic behaviour.

\medskip
Finally, we investigate some guesses on the non-existence of homology exponents for finite Postnikov towers. We conjecture that Postnikov pieces do not admit any (co)homology exponent.

\medskip
This work also includes the presentation of the ``Eilenberg-MacLane machine'', a C++ program designed to compute explicitely all integral homology groups of Eilenberg-MacLane spaces.

\selectlanguage{french}

\chapter*{R\'esum\'e}

Depuis le s\'eminaire H. Cartan de 1954-55, il est bien connu que l'on peut trouver des \'el\'ements de torsion arbitrairement grande dans l'homologie enti\`ere des espaces d'Eilenberg-MacLane $K(G,n)$ o\`u $G$ est un groupe ab\'elien non trivial et $n\geq2$. 

\medskip
L'objectif majeur de ce travail est d'\'etendre ce r\'esultat \`a des H-espaces poss\'edant plus d'un groupe d'homotopie non trivial.

\medskip
Dans le but de contr\^oler pr\'ecis\'ement le r\'esultat de H. Cartan, on commence par \'etudier la dualit\'e entre l'homologie et la cohomologie des espaces d'Eilenberg-MacLane $2$-locaux de type fini. On parvient ainsi \`a raffiner quelques r\'esultats qui d\'ecoulent des calculs de H. Cartan. 

\medskip
Le r\'esultat principal de ce travail peut \^etre formul\'e comme suit. Soit $X$ un H-espace ne poss\'edant que deux groupes d'homotopie non triviaux, tous deux finis et de $2$-torsion. Alors $X$ n'admet pas d'exposant pour son groupe gradu\'e d'homologie enti\`ere r\'eduite.

\medskip
On construit une large classe d'espaces pour laquelle ce r\'esultat n'est qu'une cons\'equence d'une caract\'eristique topologique, \`a savoir l'existence d'un r\'etract faible $X\to K(G,n)$ pour un certain groupe ab\'elien $G$ et $n\geq2$.

\medskip
On g\'en\'eralise \'egalement notre r\'esultat principal \`a des espaces plus compliqu\'es en utilisant la suite spectrale d'Eilenberg-Moore ainsi que des m\'ethodes analytiques faisant appara\^itre les nombres de Betti et leur comportement asymptotique.

\medskip
Finalement, on conjecture que les espaces qui ne poss\`edent qu'un nombre fini de groupes d'homotopie non triviaux n'admettent pas d'exposant homologique.

\medskip
Ce travail contient par ailleurs la pr\'esentation de la``machine d'Eilenberg-MacLane", un programme C++ con\c{c}u pour calculer explicitement les groupes d'homologie enti\`ere des espaces d'Eilenberg-MacLane.

\selectlanguage{english}

\endinput