\section{H. Cartan's description}

Given an associative H-space $X$, R. J. Milgram \cite{Milg66} gives a construction for a classifying space $\widehat{B} X$ which has the following advantages over that of A. Dold and R. Lashof \cite{DL59}:
\begin{itemize}
\item[1.]{if $X$ is abelian then $\widehat{B} X$ is also an abelian associative H-space with unit,}
\item[2.]{if $X$ is a CW-complex and the multiplication is cellular then $\widehat{B} X$ is also a CW-complex and the cellular chain complex $C_*(\widehat{B} X)$ is isomorphic to the bar construction $B(C_*(X))$ on the cellular chain complex,}
\item[3.]{there is an explicit diagonal approximation $\widehat{B} X\to \widehat{B} X\times \widehat{B} X$ which is cellular and, in $C_*(\widehat{B} X)$, induces exactly H. Cartan's diagonal approximation for the bar construction.}
\end{itemize}

Let $G$ be an abelian group with the discrete topology. Then it is a CW-complex consisting of $0$-cells and the multiplication is cellular. Therefore, one can inductively apply the R. J. Milgram's construction to construct $K(G,n)$ spaces for all $n\geq0$. R. J. Milgram proved the following result.

\begin{thm}\label{t:Milgram}
Let $G$ be an abelian group and $n\geq0$. The Eilenberg-MacLane space $K(G,n)$ is a topological abelian group and a CW-complex with cellular multiplication. Moreover, for all $n\geq1$, there is an isomorphism of DGA-algebras
$$
C_*(K(G,n))\cong B(C_*(K(G,n-1)))
$$ and, for any ring $R$, there are isomorphisms
\begin{align*}
H_*(K(G,n);R)&\cong H_*(B(C_*(K(G,n-1)));R),\\
H^*(K(G,n);R)&\cong H^*(B(C_*(K(G,n-1)));R),
\end{align*} being ring homomorphisms respectively of Pontrjagin and cohomology rings.
\end{thm}

\begin{proof}
See \cite[Theorems 4.1 and 4.2, p. 249]{Milg66}.
\end{proof}

Thus iterating the bar construction on $C_*(K(G,0))=\Z G$ and taking the (co)homology computes the (co)homology of Eilenberg-MacLane spaces associated to $G$. In other words we have $H_*(K(G,n);R)\cong H_*(B^n\Z G;R)$, where $B^n\Z G$ denotes the $n$-th iterated bar construction on $\Z G$. H. Cartan \cite{Ca55} studied the (co)homology of $B^n\Z G$ and therefore was able to compute the (co)homology of Eilenberg-MacLane spaces. Let us present and summarize his results in what follows.

\begin{prop}\index{0@$\sigma$, $\psi_{2^s}$, $\varphi_2$, $\gamma_2$, $\beta_2$}
Let $G$ be a non-trivial finitely generated $2$-torsion abelian group and $n\geq0$. There are maps
\begin{align*}
&\sigma:G\otimes \Z/2^s\to H_1(B\Z G;\Z/2^s)&&\text{for all $s\geq1$,}\\
&\sigma:H_q(B^n\Z G;\Z/2^s)\to H_{q+1}(B^{n+1}\Z G;\Z/2^s)&&\text{for all $q\geq0$ and $s\geq1$,}\\
&\varphi_2:H_{2q}(B^n\Z G;\F_2)\to H_{4q+2}(B^{n+1}\Z G;\F_2)&&\text{for all $q\geq1$,}\\
&\gamma_2:H_q(B^n\Z G;\F_2)\to H_{2q}(B^n\Z G;\F_2) &&\text{for all $q\geq0$,}\\
&\beta_2:H_q(B^n\Z G;\F_2)\to H_{q-1}(B^n\Z G;\F_2), &&\text{for all $q\geq0$,}
\end{align*}
verifying the basic relations\index{basic relations} (when it makes sense)
\begin{align*}
\tag{R1}\label{f:R1}&\beta_2\sigma\varphi_2=\sigma^2\gamma_2,\\
\tag{R2}\label{f:R2}&\varphi_2=\gamma_2\sigma,\\
\tag{R3}\label{f:R3}&\beta_2\sigma=\sigma\beta_2,\\
\tag{R4}\label{f:R4}&\beta_2\varphi_2=\sigma\gamma_2+\beta_2\sigma\cdot\sigma,\\
&\text{where $\cdot$ denotes the Pontryagin product.}
\end{align*}
The maps $\sigma$ and $\beta_2$ are the {\bf homology suspension}\index{homology suspension}\index{suspension} and the {\bf Bockstein} homomorphisms\index{Bockstein homomorphism} respectively. The maps $\varphi_2$ and $\gamma_2$ are called {\bf transpotence}\index{transpotence} and {\bf divided square}\index{divided square} respectively.
\end{prop}

\begin{proof}
Existence of $\sigma$, $\varphi_2$, $\gamma_2$ and $\beta_2$ is stated in \cite{Ca55}, Expos\'e 6 pp. 1-2, Expos\'e 6 Th\'eor\`eme 3 p. 8, Expos\'e 7 p. 11 and Expos\'e 8 p. 3 respectively. Basic relations \eqref{f:R2}, \eqref{f:R3} and \eqref{f:R4} are stated in Expos\'e 8, Proposition 1 p. 1, Proposition 2 p. 3 and Th\'eor\`eme 1 p. 4 respectively. Basic relation \eqref{f:R1} is a consequence of the facts that, following \eqref{f:R4}, $\beta_2\varphi_2=\sigma\gamma_2$ modulo decomposables, that the suspension $\sigma$ is trivial on decomposables (Expos\'e 6 Proposition 1 p. 2) and that, following \eqref{f:R3}, suspension commutes with Bockstein.
\end{proof}

\begin{rem*}
In what follows, we will use supplementary maps 
\begin{align*}
&\psi_{2^s}:{}_{2^s}G\to H_2(B\Z G;\Z/2^s) &&\text{for all $s\geq1$,}
\end{align*}
where ${}_{2^s}G$ is the subgroup of $G$ containing all the elements of order $\leq 2^s$ (for instance, if $G=\Z/2^r$, $r\geq2$, is generated by $u\in G$,  then $\psi_2$ is defined on ${}_{2}G$ which is generated by $u'=2^{r-1}u$). These maps are defined by H. Cartan in \cite{Ca55}, Expos\'e 11 pp. 1-2.
\end{rem*}

We will abstract some compositions of the maps $\sigma$, $\varphi_2$, $\gamma_2$, $\beta_2$ and $\psi_{2^s}$ in the notion of so called {\it admissible words}.

\begin{defn}\label{d:words}\index{letter}\index{word}\index{synonym}\index{admissible word}\index{first kind}\index{second kind}\index{degree}\index{height}\index{stable degree}\index{0@$\mathcal W$, $\mathcal W^I$, $\mathcal W^{II}$}
Let $W$ be the free monoid (with unit denoted ``$()$'') generated by the symbols $\sigma$, $\psi_{2^s}$ for all integers $s\geq1$, $\varphi_2$, $\gamma_2$ and $\beta_2$. Such a symbol is called a {\bf letter} and any element in $W$ is called a {\bf word}. Letters composing a word are read from the left to the right. On $W$ consider the equivalence relation generated by:
\begin{align*}
\tag{R1'}\label{f:R1_prime}&\beta_2\sigma\varphi_2\sim\sigma^2\gamma_2,\\
\tag{R2'}\label{f:R2_prime}&\varphi_2\sim\gamma_2\sigma,\\
\tag{R3'}\label{f:R3_prime}&\beta_2\sigma\sim\sigma\beta_2,\\
\tag{R4'}\label{f:R4_prime}&\beta_2\varphi_2\sim\sigma\gamma_2.
\end{align*}
Two words $\alpha$ and $\alpha'$ are {\bf synonyms} if $\alpha\sim\alpha'$. A word $\alpha$ is {\bf admissible} if it is a synonym of one of the following words:
\begin{itemize}
\item[$\bullet$]{$\sigma^k$ with $k\geq1$,}
\item[$\bullet$]{$\sigma^k\psi_{2^s}$ with $k\geq0$ and $s\geq1$,}
\item[$\bullet$]{$\sigma^k\varphi_2\gamma_2^{h_{i+1}}\epsilon_i\gamma_2^{h_i}\dots\epsilon_1\gamma_2^{h_1}\epsilon$ with $k\geq0$, $i\geq0$, $\epsilon_j\in\{\sigma^2,\varphi_2\}$ for all $1\leq j\leq i$ and $\epsilon\in\{\sigma^2,\psi_2\}$,}
\item[$\bullet$]{$\beta_2\sigma^k\varphi_2\gamma_2^{h_{i+1}}\epsilon_i\gamma_2^{h_i}\dots\epsilon_1\gamma_2^{h_1}\epsilon$ with $k\geq0$, $i\geq0$, $\epsilon_j\in\{\sigma^2,\varphi_2\}$ for all $1\leq j\leq i$ and $\epsilon\in\{\sigma^2,\psi_2\}$.}
\end{itemize} Let ${\mathcal W}$ be the set of all admissible words divided out by $\sim$. We will not distinguish admissible words and their classes of synonyms. An admissible word is of {\bf first kind} (resp. {\bf second kind}) if it has (a synonym with) $\sigma$ (resp. $\psi_2$) as last letter. The sets ${\mathcal W}^I$ and ${\mathcal W}^{II}$ contain admissible words of first and second kind respectively. The {\bf degree} is a map $\deg:W\to\N$ defined by induction by the following rules:
\begin{itemize}
\item[$\bullet$]{$\deg()=0$,}
\item[$\bullet$]{$\deg(\beta_2\alpha)=\deg(\alpha)-1$,}
\item[$\bullet$]{$\deg(\sigma\alpha)=\deg(\alpha)+1$,}
\item[$\bullet$]{$\deg(\gamma_2\alpha)=2\deg(\alpha)$,}
\item[$\bullet$]{$\deg(\varphi_2\alpha)=2\deg(\alpha)+2$ and}
\item[$\bullet$]{$\deg(\psi_{2^s})=2$ for all $s\geq1$.}
\end{itemize}
The degree induces a {\bf grading} on the set $\mathcal W$ of all admissible words. The {\bf height} is a map $h:W\to\N$ which maps every word $\alpha\in W$ to the number of its letters equals to $\sigma$, $\varphi_2$ or $\psi_{2^s}$, $s\geq1$. Inductively we have:
\begin{itemize}
\item[$\bullet$]{$h()=0$,}
\item[$\bullet$]{$h(\beta_2\alpha)=h(\alpha)$,}
\item[$\bullet$]{$h(\sigma\alpha)=h(\alpha)+1$,}
\item[$\bullet$]{$h(\gamma_2\alpha)=h(\alpha)$,}
\item[$\bullet$]{$h(\varphi_2\alpha)=h(\alpha)+1$ and}
\item[$\bullet$]{$h(\psi_{2^s})=1$ for all $s\geq1$.}
\end{itemize} The height induces a map on the set $\mathcal W$ of all admissible words. The {\bf stable degree} is a map $\degst:{\mathcal W}\to\N$ defined by $\degst(\alpha)=\deg(\alpha)-h(\alpha)$ for all $\alpha\in{\mathcal W}$. Let ${\mathcal W}_{q,n}$ be the graded subset given by all admissible words $\alpha\in{\mathcal W}$ such that $\degst(\alpha)=q$ and $h(\alpha)=n$. Let ${\mathcal W}_{*,n}=\cup_{q\geq0}{\mathcal W}_{q,n}$. We also define ${\mathcal W}^I_{q,n}={\mathcal W}_{q,n}\cap{\mathcal W}^I$, ${\mathcal W}^{II}_{q,n}={\mathcal W}_{q,n}\cap{\mathcal W}^{II}$, ${\mathcal W}^I_{*,n}={\mathcal W}_{*,n}\cap{\mathcal W}^I$ and ${\mathcal W}^{II}_{*,n}={\mathcal W}_{*,n}\cap{\mathcal W}^{II}$.
\end{defn}

It may be useful to make some remarks.
\begin{itemize}
\item[1.]{First of all, it is very easy to verify that the degree, height and stable degree agree on synonyms. It suffices to see that $\deg(\beta_2\sigma\varphi_2\alpha)=2\deg(\alpha)+2=\deg(\sigma^2\gamma_2\alpha)$, $\deg(\varphi_2\alpha)=2\deg(\alpha)+2=\deg(\gamma_2\sigma\alpha)$, $\deg(\beta_2\sigma\alpha)=\deg(\alpha)=\deg(\sigma\beta_2\alpha)$, $\deg(\beta_2\varphi_2\alpha)=2\deg(\alpha)+1=\deg(\sigma\gamma_2\alpha)$, $h(\beta_2\sigma\varphi_2\alpha)=h(\alpha)+2=h(\sigma^2\gamma_2\alpha)$, $h(\varphi_2\alpha)=h(\alpha)+1=h(\gamma_2\sigma\alpha)$, $h(\beta_2\sigma\alpha)=h(\alpha)+1=h(\sigma\beta_2\alpha)$ and $h(\beta_2\varphi_2\alpha)=h(\alpha)+1=h(\sigma\gamma_2\alpha)$.}
\medskip

\item[2.]{The height of an admissible word $\alpha$ cannot be zero since $\alpha$ is non-empty and its last letter is $\sigma$ or $\psi_{2^s}$, $s\geq1$, all of height equal to $1$.}
\medskip

\item[3.]{The equivalence given by \eqref{f:R4_prime} does not agree with the basic relation \eqref{f:R4}. The main reason is that we cannot consider any object like $\sigma\gamma_2+\beta_2\sigma\cdot\sigma$ in the monoid $W$. Let us see why we persist to set the relation \eqref{f:R4_prime}. Assume that the set $\{\beta_2\varphi_2\alpha,\sigma\gamma_2\alpha,\beta_2\sigma\alpha\cdot\sigma\alpha\}$ plays the role of an $\F_2$-generator set in the computation of the graded $\F_2$-vector space $H_*(K(G,n);\F_2)$ (this is actually the case by Cartan's results). The generator $\sigma\gamma_2\alpha$ is a linear combination of $\beta_2\varphi_2\alpha$ and $\beta_2\sigma\alpha\cdot\sigma\alpha$ since basic relation \eqref{f:R4} holds. Therefore the two generators $\beta_2\varphi_2\alpha$ and $\beta_2\sigma\alpha\cdot\sigma\alpha$ form a basis. Suppose now in $W$ that we do not set any relation between the three elements $\beta_2\varphi_2\alpha$, $\beta_2\sigma\alpha\cdot\sigma\alpha$ and $\sigma\gamma_2\alpha$. In this case, we would have three basis elements. Thus it is important to increase the number of relations by one in order to decrease the dimension to two. The only possible relations we could set are $\beta_2\varphi_2\sim\sigma\gamma_2$, $\beta_2\varphi_2\sim\beta_2\sigma\cdot\sigma$ and $\beta_2\varphi_2\sim\sigma\gamma_2+\beta_2\sigma\cdot\sigma$. The only formally possible one in the monoid $W$ is $\beta_2\varphi_2\sim\sigma\gamma_2$, namely \eqref{f:R4_prime}. Relation \eqref{f:R4_prime} mimics \eqref{f:R4} in the sense that the number of relations between elements of given degree and height is preserved.}
\medskip

\item[4.]{For all admissible word $\alpha\in{\mathcal W}$, one can obviously associate a map $\alpha:G\otimes\Z/2^s\to H_*(K(G,n);\F_2)$ if $\alpha\in{\mathcal W}^I$ or $\alpha:{}_{2^s}G\to H_*(K(G,n);\F_2)$ if $\alpha\in{\mathcal W}^{II}$ (in the case where $\alpha=\beta_2\varphi_2\alpha'$ or $\sigma\gamma_2\alpha'$, the associated map is $\beta_2\varphi_2\alpha'$). Moreover, this map is linear when $G$ is a cyclic group. See \cite[Expos\'e 9, pp. 1-2; Proposition 2, p. 2 and Derni\`ere remarque, p.10]{Ca55}. We will not distinguish admissible words and their corresponding linear maps in what follows.}
\end{itemize}
\bigskip

\vskip5cm
\begin{defn}\index{divided power algebra}
Let $A$ be a graded algebra which is commutative (in the graded sense). We say that $A$ is endowed with a {\bf divided powers} system if, for each $x\in A$ of positive even degree $\deg(x)$, there is a sequence of elements $\gamma_i(x)\in A$ for all $i\geq0$, in degrees $\deg(\gamma_i(x))=i\deg(x)$, satisfying the following properties:
\begin{align*}
\tag{1}\label{eq:gamma_0}&\gamma_0(x)=1,\\
\tag{2}&\gamma_1(x)=x,\\
\tag{3}\label{eq:divided powers}&\text{$\gamma_i(x)\gamma_j(x)=\binom{i+j}{i}\gamma_{i+j}(x)$,}\\
\tag{4}&\gamma_i(x+y)=\sum_{i=k+l}\gamma_k(x)\gamma_l(y),\\
\tag{5}\label{eq:gamma on product}&\text{for all $i\geq0$, $\gamma_i(xy)=\begin{cases}
0 &\text{if $\deg(x)$, $\deg(y)$ are odd, $i\geq2$}\\
x^i\gamma_i(y) &\text{if $\deg(x)$, $\deg(y)\geq2$ are even,}
\end{cases}$}\\
\tag{6}\label{eq:gamma product}&\gamma_j(\gamma_i(x))=\prod_{k=i}^{(j-1)i}\binom{k+i-1}{k}\gamma_{ij}(x).
\end{align*}
Property \eqref{eq:divided powers} implies that $x^i=i!\gamma_i(x)$ for all $i\geq0$, which justifies the terminology of {\it divided powers} for the elements $\gamma_i(x)$. We will often omit the parenthesis in the notation $\gamma_i(x)$ and simply write $\gamma_ix$.
\end{defn}

Let us make some remarks on divided powers algebras in characteristic $2$ that will be useful in the sequel. In this case, divided powers are called {\bf divided squares}\index{divided square}. Property \eqref{eq:gamma product} gives:
\begin{align*}
\tag{6'}\label{eq:gamma_2 product}\gamma_i(\gamma_2(x))=\gamma_{2i}(x).
\end{align*}
Moreover, if one consider {\bf strictly commutative}\index{strictly commutative} graded algebras $A$ in characteristic $2$, i.e. commutative algebras such that $x^2=0$ if $\deg(x)>0$, one can define $\gamma_i(x)$ on every element $x\in A$ such that $\deg(x)\geq2$ (even or odd). In this case, properties \eqref{eq:gamma_0} to \eqref{eq:gamma product} remains valid and property \eqref{eq:gamma on product} becomes:
\begin{align*}
\tag{5'}&\text{for all $i\geq2$, $\gamma_i(xy)=\begin{cases}
0 &\text{if $\deg(x)$, $\deg(y)$ are $>0$,}\\
x^i\gamma_i(y) &\text{if $\deg(x)=0$.}
\end{cases}$}
\end{align*}

Let us look at a very important exemple of divided powers algebra. The {\bf divided polynomial algebra}\index{divided polynomial algebra} $\Gamma_R[x]$ on one generator of even degree $q$ is defined as a graded $R$-module having generators $x_i$, in degrees $qi$, for all $i\geq0$, with the following properties:
\begin{align*}
x_0&=1,\\
x_1&=x,\\
x_i x_j&=\binom{i+j}{i}x_{i+j} &&\text{for all $i,j\geq0$.}
\end{align*}
This algebra is endowed with a divided power algebra system given by
\begin{align*}
\gamma_0(x_i)&=1,\\
\gamma_j(x_i)&=\prod_{k=i}^{(j-1)i}\binom{k+i-1}{k}x_{ij} &&\text{for all $i\geq0$ and $j\geq1$.}
\end{align*}
In particular, we have $\gamma_j(x)=x_j$ for all $j\geq0$. Suppose that we are working in characteristic $2$. Let $j=j_0+j_12+\dots+j_k2^k$ be the $2$-adic developpment of $j$, with $j_0,\dots,j_k\in\{0,1\}$. Then property \eqref{eq:gamma_2 product} implies that
\begin{align*}
\gamma_j(x)&=\underbrace{\gamma_{j_0}(x)}_{\in\{1,x\}}\underbrace{\gamma_{j_1}(\gamma_2(x))}_{\in\{1,\gamma_2(x)\}}\dots\underbrace{\gamma_{j_k}(\gamma_{2^k}(x))}_{\in\{1,\gamma_{2^k}(x)\}}\\
&=\gamma_{j_0}(x)\ \gamma_{j_1}(\gamma_2(x))\ \gamma_{j_2}(\gamma_2\gamma_2(x))\dots\gamma_{j_k}(\underbrace{\gamma_2\dots\gamma_2}_{\text{$k$ times}}(x))\\
&=\gamma_{j_0}(x)\ \gamma_{j_1}(\gamma_2(x))\ \gamma_{j_2}(\gamma_2^2(x))\dots\gamma_{j_k}(\gamma_2^k(x)).
\end{align*}
It is then easy to see that the following isomorphisms hold:
\begin{align*}
\Gamma_{\F_2}[x]&\cong\bigotimes_{i\geq0}\Lambda_{\F_2}(\gamma_2^i x) &&\text{as graded $\F_2$-algebras,}\\
&\cong\F_2[x] &&\text{as graded $\F_2$-vector spaces,}
\end{align*} with $\gamma_2^i x\mapsto x^{2^i}$ for all $i\geq0$.

The {\bf exterior algebra}\index{exterior algebra} $\Lambda_R(x)$ on one generator of degree $q$ is the graded quotient algebra $R[x]/(x^2)$ of the polynomial algebra $R[x]$. We will write $\Gamma[x]$ and $\Lambda(x)$ instead of $\Gamma_{\Z}[x]$ and $\Lambda_{\Z}(x)$ respectively.

\begin{defn}\index{elementary complex}
Let $(x,y)$ be a couple of positive bidegree \hbox{$(q,q+1)$}. For all $h\in\Z$ we define the {\bf elementary complex} associated to $(x,y)$ as follows:
\begin{align*}
&EC_h(x,y)=\begin{cases}
(\Gamma[x]\otimes \Lambda(y),d_h) &\text{if $q$ is even,}\\
(\Lambda(x)\otimes \Gamma[y],d_h) &\text{if $q$ is odd,}
\end{cases}
\end{align*}
with differential $d_h$ given in both cases by
\begin{align*}
&d_h x=0\ \text{and}\\ 
&d_h y=hx.
\end{align*}
\end{defn}

\begin{lem}\label{l:homology of EC}
Let $(x,y)$ be a couple of positive bidegree $(q,q+1)$ and $h\in\Z$. Then we have
$$
H_n(EC_h(x,y);\Z)\cong
$$
$$
\begin{cases}
\Z/|\ell h|<\gamma_\ell(x)> &\text{if $q$ is even and $n=\ell q$}\\
\Z/|h|<x\gamma_\ell(y)> &\text{if $q$ is odd and $n=(\ell+1)q+\ell$}\\
0&\text{otherwise.}
\end{cases}
$$
\end{lem}

\begin{proof}
The proof is immediate from the definitions. See \cite[Expos\'e 11, p. 3]{Ca55}.
\end{proof}

\begin{rem}\label{r:high order classes}
If $x$ is of even degree $q\geq2$ and $h\not=0$, then there are homogeneous classes of {\bf arbitrarily high order} in the integral homology of $EC_h(x,y)$, namely those of order $|\ell h|$ given by $\gamma_\ell(x)$ in dimensions $\ell q$ for all $\ell\geq1$.
\end{rem}

\begin{defn}\index{Cartan's complex}\label{d:generators}
Let $G$ be a non-trivial finitely generated $2$-torsion abelian group of type $(s_1,\dots,s_l)$ and $n\geq1$. Let $U=\{u_1,\dots,u_l\}$ be a set of generators for $G$ such that $u_j$ is of order $s_j$ for all $1\leq j\leq l$. For all $n\geq1$ we define the following complexes:
\begin{align*}
(X',d')&=\bigotimes_{1\leq j\leq l}EC_{(-1)^{n-1}2^{s_j}}(\sigma^n u_j,\sigma^{n-1}\psi_{2^{s_j}} u_j),\\
(X'',d'')&=\bigotimes_{\substack{1\leq j\leq l\\ 0\leq k\leq n-3\\ \alpha\in{\mathcal W^I_{*,n-k-1}}}}EC_{(-1)^k 2}(\beta_2\sigma^k\varphi_2\alpha u_j,\sigma^k\varphi_2\alpha u_j),\\
(X''',d''')&=\bigotimes_{\substack{1\leq j\leq l\\ 0\leq k\leq n-2\\ \alpha\in{\mathcal W^{II}_{*,n-k-1}}}}EC_{(-1)^k 2}(\beta_2\sigma^k\varphi_2\alpha u'_j,\sigma^k\varphi_2\alpha u'_j),\\ 
&\text{where $u'_j=2^{s_j-1}u_j$ for all $1\leq j\leq l$.}
\end{align*}
Generators of $X'$, $X''$ and $X'''$ are of {\bf genus}\index{genus} $1$, $2$ and $3$ respectively. The tensor product $(X,d)=(X',d')\otimes (X'',d'')\otimes (X''',d''')\otimes\Z_{(2)}$ is the {\bf Cartan's complex} associated to $G$ and $n$.
\end{defn}

\begin{rem}
This definition is {\it ad hoc} for our computations at the prime $2$. Actually, H. Cartan gives a much more general one for all primes. See \cite[Expos\'e 11, pp.5-7]{Ca55} or \cite[D\'efinition 2.6, pp. 32-33]{Po96} for a complete and original definiton.
\end{rem}

\begin{thm}[H. Cartan, 1955]\label{t:Cartan}
Let $G$ be a non-trivial finitely generated $2$-torsion abelian group, $n\geq1$ and $X$ the associated Cartan's complex. There is an isomorphism of complexes 
$$B^n \Z G\cong X$$
and, for any ring $R$, isomorphisms of graded $R$-modules
\begin{align*}
H_*(B^n\Z G;R)&\cong H_*(X;R),\\
H^*(B^n\Z G;R)&\cong H^*(X;R).
\end{align*}
\end{thm}

\begin{proof}
See \cite[Expos\'e 11, pp. 7-10]{Ca55} and, specifically when $R$ is of characteristic $2$, see \cite[Expos\'e 9, Derni\`ere remarque, p. 10]{Ca55}.
\end{proof}

\newpage
\begin{cor}
Let $G$ be a non-trivial finitely generated $2$-torsion abelian group, $n\geq1$ and $X$ the associated Cartan's complex. For any ring $R$, there are isomorphisms of graded $R$-modules
\begin{align*}
H_*(K(G,n);R)&\cong H_*(X;R)\text{ and}\\
H^*(K(G,n);R)&\cong H^*(X;R).
\end{align*}
\end{cor}

\begin{proof}
We have
\begin{align*}
H_*(K(G,n);R) &\cong H_*(B^n(C_*(K(G,0)))) &&\text{by Theorem \ref{t:Milgram}}\\
&\cong H_*(B^n\Z G;R)\\
&\cong H_*(X;R) &&\text{by Theorem \ref{t:Cartan},}
\end{align*} the last isomorphism being an isomorphism of graded $R$-modules.
\end{proof}

\begin{thm}\label{t:Cartan's description in homology}
Let $n\geq1$ and $s\geq1$. The graded $\F_2$-vector space $H_*(K(\Z/2^s,n);\F_2)$ is isomorphic to
$$
\Gamma_{\F_2}[{\mathcal W}_{*,n}^+]\otimes\Lambda_{\F_2}({\mathcal W}_{*,n}^-),
$$ where ${\mathcal W}_{*,n}^+$ and ${\mathcal W}_{*,n}^-$ denote the set of all even, respectively odd degree elements in ${\mathcal W}_{*,n}$.
\end{thm}

\begin{proof}
It is obvious to see that the complex $X\otimes\Z/2$ is acyclic. Therefore, its homology is given by $X\otimes\Z/2$ itself. Let us compute $EC_0(x,y)\otimes\F_2$. If $x$ is of even degree, then 
$$EC_0(x,y)\otimes\F_2=(\Gamma[x]\otimes \Lambda(y))\otimes\F_2\cong\Gamma_{\F_2}[x]\otimes\Lambda_{\F_2}(y)$$ 
as graded $\F_2$-vector spaces. A similar result holds when $x$ is of odd degree. To conclude, it suffices now to see that 
\begin{align*}
{\mathcal W}_{*,n}=&\{\sigma^n,\sigma^{n-1}\psi_{2^s}\}\ \cup\\
&\bigcup_{\substack{0\leq k\leq n-3\\ \alpha\in{\mathcal W^I_{*,n-k-1}}}}\{\beta_2\sigma^k\varphi_2\alpha,\sigma^k\varphi_2\alpha\}\ \cup\\
&\bigcup_{\substack{0\leq k\leq n-2\\ \alpha\in{\mathcal W^{II}_{*,n-k-1}}}}\{\beta_2\sigma^k\varphi_2\alpha,\sigma^k\varphi_2\alpha\}.
\end{align*}
\end{proof}

A very similar proof gives the following dual result in cohomology:

\begin{thm}\label{t:Cartan's description in cohomology}
Let $n\geq1$ and $s\geq1$. The graded $\F_2$-vector space $H^*(K(\Z/2^s,n);\F_2)$ is isomorphic to
$$
\F_2[{\mathcal W}_{*,n}^+]\otimes\Lambda_{\F_2}({\mathcal W}_{*,n}^-).
$$
\end{thm}

\begin{cor}\label{c:Hopf structure of K(G,n) in homology}
Let $n\geq1$ and $s\geq1$. The differential graded $\F_2$-algebra $H_*(K(\Z/2^s,n);\F_2)$ is the connected, biassociative, bicommutative and coprimitive differential graded Hopf algebra given by $\Gamma_{\F_2}[\mathcal{W}_{*,n}^+]\otimes\Lambda_{\F_2}(\mathcal{W}_{*,n}^-)$.
\end{cor}

\begin{proof}
It suffices to remark that $\Gamma_{\F_2}[x]^\text{dual}\cong\F_2[x]$ as Hopf algebras and to consider Theorem \ref{t:Hopf structure of K(G,n) cohomology}.
\end{proof}

\begin{exmp}
Since ${\mathcal W}_{*,1}=\{\sigma,\psi_2\}$ with $\deg(\sigma)=1$ and $\deg(\psi_2)=2$ we have
$$
H^*(K(\Z/2,1);\F_2)\cong\F_2[\psi_2]\otimes\Lambda_{\F_2}(\sigma).
$$
It is easy to verify that we have an isomorphism of graded $\F_2$-vector spaces
$$
\F_2[\psi_2]\otimes\Lambda_{\F_2}(\sigma)\cong\F_2[u_1],
$$ where $\F_2[u_1]$ is the Serre's description of the cohomology algebra (see Theorem \ref{t:Serre's description}), given by
\begin{align*}
\psi_2^k\otimes1 &\mapsto u_1^{2k},\\
\psi_2^k\otimes\sigma &\mapsto u_1^{2k+1},
\end{align*} for all $k\geq0$.
\end{exmp}

\endinput

\begin{rem*}
It is important to keep in mind that $x^{2^r}=(2^r)!\gamma_2^r x$. Therefore, if $x\in{\mathcal W}^+_{*,n}$, then $x^{2^r}=0\in H_*(K(\Z/2^s,n);\F_2)$ or $H^*(K(\Z/2^s,n);\F_2)$. But if we identify $\F_2[x]$ with $\Gamma_{\F_2}[x]$ as vector spaces, then $x^{2^r}$ corresponds to $\gamma_2^r x$, which is not trivial.
\end{rem*}

\endinput
%%%

\begin{rem}
Rappelons qu'en caract\'eristique $2$ l'alg\`ebre commutative gradu\'ee libre $FGCA(X)$ sur un ensemble gradu\'e $X$ est isomorphe \`a l'alg\`ebre de polyn�mes $\F_2[X]$. Le th\'eor\`eme \ref{t:Cartan_mod_2} expos\'e ici n'est pas celui auquel on est accoutum\'e puisque il n'exprime pas $H^*(K(\Z/2^f,n);\F_2))$ en termes d'alg\`ebre commutative gradu\'ee libre, comme le sugg\`ere le r\'esultat de J.-P. Serre. Il s'agit en fait du r\'esultat fondamental de l'expos\'e 9 du s\'eminaire Cartan, originellement \'etabli en caract\'eristique impaire. Toutefois, lorsqu'il s'agit de groupes cycliques, comme c'est le cas ici avec $\Z/2^f$, on peut montrer que ce r\'esultat vaut en caract\'eristique $2$ si l'on se restreint \`a la structure de $\F_2$-espace vectoriel. Remarquons que H. Cartan donne une description de l'alg\`ebre $H^*(K(\Z/2^f,n);\F_2))$ comme alg\`ebre commutative gradu\'ee libre dans l'expos\'e 10. Cette description n\'ecessite l'usage d'un jeu de mots admissibles diff\'erent de celui utilis\'e ici et dans les expos\'es 9 et 11. Or l'expos\'e 11 calcule la cohomologie enti\`ere des espace d'Eilenberg-MacLane et on sera bient�t amen\'es \`a calculer l'induite de l'application canonique $\red_2:\Z\to\Z/2$ induite en cohomologie. Il sera donc avantageux d'avoir un jeu de mots admissibles commun pour la cohomologie enti\`ere et mod $2$. Ceci explique que l'on pr\'ef\`ere la formulation du th\'eor\`eme \ref{t:Cartan_mod_2} \`a celle du th\'eor\`eme de l'expos\'e 10.
\end{rem}

\begin{exmp}
D'apr\`es le th\'eor\`eme de H. Cartan on a 
$$
H^*(K(\Z/2,1);\F_2)\cong\Lambda_{\F_2}({\mathcal W}_{*,1}^-)\otimes_{\F_2} \F_2[{\mathcal W}_{*,1}^+].
$$
Or on v\'erifie facilement que 
$$
{\mathcal W}_{*,1}=\{\sigma, \psi_2\},
$$ avec $\sigma$ de degr\'e $1$ et $\psi_2$ de degr\'e $2$.
Ainsi on a 
$$
H^*(K(\Z/2,1);\F_2)\cong\Lambda_{\F_2}(\{\sigma\})\otimes_{\F_2} \F_2[\psi_2]\cong\F_2[u_1],
$$ o\`u $u_1\in H^1(K(\Z/2,1);\F_2)$ d\'esigne la classe caract\'eristique et le dernier isomorphisme de $\F_2$-espaces vectoriels est donn\'e par 
$$1\otimes\psi_2^k\mapsto u_1^{2k}\text{ et}$$
$$\sigma\otimes\psi_2^k\mapsto u_1^{2k+1}$$ pour tout $k\geq0$. Remarquons qu'on n'a pas d'isomorphisme d'alg\`ebres: si c'\'etait le cas, on aurait obligatoirement $\sigma\otimes1\mapsto u_1$ et donc $0=(\sigma\otimes1)^2=\sigma^2\otimes1\mapsto u_1^2\not=0$, ce qui est absurde.
\end{exmp}

\endinput
%%%%%%%%%%%%%%%%%%%%%%%%%%%%%%%
Let $\sim$ be the following equivalence relation on $W$:
\begin{align*}
\tag{R1'}\label{f:R1prime}&\sigma^k\psi_{2^s}\sim\sigma^k\varphi_2 \text{ for all $s\geq2$},\\
\tag{R1''}\label{f:R1second}&\alpha\psi_2\sim\alpha\varphi_2,\\
\tag{R2'}\label{f:R2prime}&\beta_2\sigma^k\psi_{2^s}\sim\begin{cases}
\sigma^{k+1} &\text{if $s=1$,}\\
0 &\text{if $s\geq2$,}
\end{cases}\\
\tag{R2''}\label{f:R2second}&\beta_2\sigma^k\varphi_2\alpha\sim\sigma^{k+1}\gamma_2\alpha,
\end{align*}
for all $k\geq1$ and for all $\alpha\in W$ composed with the letters $\sigma$, $\gamma_2$ or $\varphi_2$. A {\bf word} is an element of the quotient set $W/\sim$. We will denote a word in the same way as one of his representant. We will privilegiate the writting of words $\beta_2\sigma^k\varphi_2\alpha$ and $\alpha\psi_2$ instead of $\sigma^{k+1}\gamma_2\alpha$ and $\alpha\varphi_2$.
\end{defn}

\begin{rem}\label{r:priority}
Relations \eqref{f:R1prime} and \eqref{f:R1second} are not compatible. For instance, if $\alpha=\sigma^k$, \eqref{f:R1second} gives $\sigma^k\psi_2\sim\sigma^k\varphi_2$ and thus $\sigma^k\psi_2\sim\sigma^k\psi_{2^s}$. But we do not want this relation to hold for all $s\geq2$. Similarly, relations \eqref{f:R2prime} and \eqref{f:R2second} are also not compatible. Therefore we fix {\bf priority on relations \eqref{f:R1prime} and \eqref{f:R2prime}}. This implies that every time we are writting a word, we have to use basic relations only after having wrote the entire word, and following these priorities.
\end{rem}