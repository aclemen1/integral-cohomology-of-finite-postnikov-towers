\chapter{Introduction}

Let $X$ be a connected space. One can both consider its graded homotopy group, $\pi_*(X)$, and its graded reduced integral homology group, $\widetilde{H}_*(X;\Z)$. If there exists an integer $h\geq1$ such that $h\cdot\pi_*(X)=0$, we then say that $X$ has a {\it homotopy exponent}. Analogously, if there exists an integer $e\geq1$ such that $e\cdot\widetilde{H}_*(X;\Z)=0$, we then say that $X$ has a {\it homology exponent}. 

The general problem posed by D. Arlettaz is to know whether there is a relationship between homotopy exponents and homology exponents. For instance, is it true that a space with a homotopy exponent has a homology exponent too? In this case, how are these two exponents related, if they are? Or conversely, is it possible for a space without a homotopy exponent to admit a homology exponent?

In the present work, we focus on spaces with homotopy exponents and ask if they have a homology exponent.

The first example of such a space which come to mind is an Eilenberg-MacLane space $K(G,n)$, with $G$ a finite group, abelian if $n\geq2$. 

It is very well known that if $G$ is a finite group, then $K(G,1)$ has a homology exponent. Actually we can prove, by using a transfer argument, that $\card(G)\cdot\widetilde{H}^*(K(G,1);\Z)=0$.

By H. Cartan's work \cite{Ca55}, it is well known that $K(G,n)$ has no homology exponent if $n\geq2$. This result gives a drastic answer to D. Arlettaz's question: the existence of a homotopy exponent does not imply the existence of a homology exponent in general. But is it always the case for $1$-connected spaces?

For instance, and as an obvious consequence of the K\"unneth formula, products of Eilenberg-MacLane spaces, which are called {\it generalized Eilenberg-MacLane spaces} (GEM), neither have a homology exponent.

The purpose of this work is then to investigate the integral (co)homology of spaces which have a finite Postnikov tower, i.e. spaces with finitely many non-trivial homotopy groups (these spaces are also called {\it Postnikov pieces}), and which are not GEM's.

Our main results (see Section \ref{section:main results}) deal with spaces with two non-trivial homotopy groups and with stable $2$-stage Postnikov systems (see p. \pageref{d:two-stage Postnikov system} for a definition). These spaces turn out not to have a homology exponent, although they have a homotopy exponent.

In order to work in an affordable and manageable technical framework, we will only compute at the prime $2$. For simplicity, our results then focus on $2$-local Postnikov pieces of finite type, and most of the time with an H-space structure.
\newpage

\input chap1sect1.tex
\newpage
\input chap1sect2.tex
\newpage
\input chap1sect3.tex
\newpage
\input chap1sect4.tex
\newpage
\input chap1sect5.tex