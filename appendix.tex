{
\appendix

\numberwithin{equation}{chapter}

%\swapnumbers
\theoremstyle{plain}
\newtheorem{Athm}{Theorem}[chapter]
\newtheorem*{Athm*}{Theorem}
\newtheorem{Alem}[Athm]{Lemma}
\newtheorem{Aprop}[Athm]{Proposition}
\newtheorem{Acor}[Athm]{Corollary}
\newtheorem*{Acor*}{Corollary}

\theoremstyle{definition}
\newtheorem{Adefn}{Definition}[chapter]
\newtheorem{Anota}{Notation}[chapter]
\newtheorem{Aconv}{Convention}[chapter]
\newtheorem{Aconj}{Conjecture}[chapter]
\newtheorem{Aexmp}{Example}[chapter]

\theoremstyle{remark}
\newtheorem*{Arem}{Remark}
\newtheorem*{Anote}{Note}
%%%%%%%%%%%%%%%%%%%%%%%%%%%%%%%%
\chapter{Adem relations among Steenrod Squares}\label{a:Adem relations}

\begin{align*}
Sq^iSq^j=\sum_{k=0}^{[i/2]}\binom{j-k-1}{i-2j}Sq^{i+j-k}Sq^k\\
\text{for all $0<i<2j$.}
\end{align*}

\bigskip
\begin{align*}
Sq^1Sq^1=0, Sq^1Sq^3=0,\dots; &&&Sq^1Sq^{2n+1}=0\\
Sq^1Sq^2=Sq^3,Sq^1Sq^4=Sq^5,\dots; &&&Sq^1Sq^{2n}=Sq^{2n+1}\\
Sq^2Sq^2=Sq^3Sq^1, Sq^2Sq^6=Sq^7Sq^1,\dots; &&&Sq^2Sq^{4n-2}=Sq^{4n-1}Sq^1\\
Sq^2Sq^3=Sq^5+Sq^4Sq^1,\dots; &&&Sq^2Sq^{4n-1}=Sq^{4n+1}+Sq^{4n}Sq^1\\
Sq^2Sq^4=Sq^6+Sq^5Sq^1,\dots; &&&Sq^{2}Sq^{4n}=Sq^{4n+2}+Sq^{4n+1}Sq^1\\
Sq^2Sq^5=Sq^6Sq^1,\dots; &&&Sq^2Sq^{4n+1}=Sq^{4n+2}Sq^1\\
Sq^3Sq^2=0, Sq^3Sq^6=0,\dots; &&&Sq^3Sq^{4n+2}=0\\
Sq^3Sq^3=Sq^5Sq^1,\dots; &&&\dots\\
\dots;&&&Sq^{2n-1}Sq^n=0\\
\vdots
\end{align*}
%%%%%%%%%%%%%%%%%%%%%%%%%%%%%%%%%
\input appendix1.tex
%%%%%%%%%%%%%%%%%%%%%%%%%%%%%%%%%
\chapter{Tables for Eilenberg-MacLane spaces homology}\label{a:tables}

The {\bf genus}\index{genus} of a generator $(x,y)$ (see Definition \ref{d:generators}, p. \pageref{d:generators}) is given by
$$
\begin{cases}
1 &\text{if $x$ is of the form $\sigma^n$,}\\
2 &\text{if $x$ is not of the form $\sigma^n$ and $x$ (or $y$) ends with $\sigma$,}\\
3 &\text{if $x$ (or $y$) ends with $\psi_2$.}\\
\end{cases}
$$

\input table1.tex
\newpage
\input table2.tex
\newpage
\input table3.tex
\newpage
\input table4.tex
%%%%%%%%%%%%%%%%%%%%%%%%%%%%%%%%%%%%
\chapter{C++ implementation of the\\ ``Eilenberg-MacLane machine''}\label{a:C++ implementation}

\renewcommand\verbatimtabsize{4\relax}

\section{Header file \texttt{Mod2GradedGroups.hpp}}
{\small
\listinginput[1]{1}{"../Eilenberg-MacLane Machine/Mod2GradedGroups.hpp"}
}

\newpage
\section{Header file \texttt{Mod2AdmissibleSequence.hpp}}
{\small
\listinginput[1]{1}{"../Eilenberg-MacLane Machine/Mod2AdmissibleSequences.hpp"}
}

\newpage
\section{Header file \texttt{Mod2Words.hpp}}
{\small
\listinginput[1]{1}{"../Eilenberg-MacLane Machine/Mod2Words.hpp"}
}

\newpage
\section{Header file \texttt{Mod2ElementaryComplexHomology.hpp}}
{\small
\listinginput[1]{1}{"../Eilenberg-MacLane Machine/Mod2ElementaryComplexHomology.hpp"}
}

\newpage
\section{Header file \texttt{Mod2EilenbergMacLaneHomology.hpp}}
{\small
\listinginput[1]{1}{"../Eilenberg-MacLane Machine/Mod2EilenbergMacLaneHomology.hpp"}
}

\newpage
\section{Header file \texttt{LaTeXLayout.hpp}}
{\small
\listinginput[1]{1}{"../Eilenberg-MacLane Machine/LaTeXLayout.hpp"}
}

}


\endinput